\tikzstyle{block} = [draw, rectangle, 
    minimum height=1.25em, minimum width=2.5em]
\tikzstyle{sum} = [draw, circle, node distance=1cm]
\tikzstyle{input} = [coordinate]
\tikzstyle{output} = [coordinate]
\tikzstyle{pinstyle} = [pin edge={to-,thin,black}]

% The block diagram code is probably more verbose than necessary
\begin{tikzpicture}[auto, node distance=2.5cm,>=latex']
    % We start by placing the blocks
    \node [input, name=input] {};
    \node [sum, right of=input] (sum) {};
    \node [block, right of=sum] (controller) {$G$};
    
    % We draw an edge between the controller and system block to 
    % calculate the coordinate u. We need it to place the measurement block. 
   
    \node [output, right of=controller] (output) {};
    \node [block, below of=controller] (measurements) {$H$};

    % Once the nodes are placed, connecting them is easy. 
    \draw [draw,->] (input) -- node[pos=0.99] {$+$} node {$v_{s}$} (sum);
    \draw [->] (sum) -- node {$v_{i}$} (controller);
    \draw [->] (controller) -- node [name=y] {$v_{o}$}(output);
    \draw [->] (y) |- (measurements);
    \draw [->] (measurements) -| node[pos=0.99] {$-$} node [near end] {$v_{f}$} (sum);
\end{tikzpicture}
