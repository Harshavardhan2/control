\documentclass[journal,12pt,twocolumn]{IEEEtran}
%
\usepackage{setspace}
\usepackage{gensymb}
%\doublespacing
\singlespacing

%\usepackage{graphicx}
%\usepackage{amssymb}
%\usepackage{relsize}
\usepackage[cmex10]{amsmath}
%\usepackage{amsthm}
%\interdisplaylinepenalty=2500
%\savesymbol{iint}
%\usepackage{txfonts}
%\restoresymbol{TXF}{iint}
%\usepackage{wasysym}
\usepackage{amsthm}
%\usepackage{iithtlc}
\usepackage{mathrsfs}
\usepackage{txfonts}
\usepackage{stfloats}
\usepackage{bm}
\usepackage{cite}
\usepackage{cases}
\usepackage{subfig}
%\usepackage{xtab}
\usepackage{longtable}
\usepackage{multirow}
%\usepackage{algorithm}
%\usepackage{algpseudocode}
\usepackage{enumitem}
\usepackage{mathtools}
\usepackage{steinmetz}
\usepackage{tikz}
\usepackage{circuitikz}
\usepackage{verbatim}
\usepackage{tfrupee}
\usepackage[breaklinks=true]{hyperref}
%\usepackage{stmaryrd}
\usepackage{tkz-euclide} % loads  TikZ and tkz-base
\usetkzobj{all}
\usetikzlibrary{decorations.markings}
\usetikzlibrary{shapes.geometric}
\newif\iflabrev
\usepackage{listings}
    \usepackage{color}                                            %%
    \usepackage{array}                                            %%
    \usepackage{longtable}                                        %%
    \usepackage{calc}                                             %%
    \usepackage{multirow}                                         %%
    \usepackage{hhline}                                           %%
    \usepackage{ifthen}                                           %%
  %optionally (for landscape tables embedded in another document): %%
    \usepackage{lscape}     
\usepackage{multicol}
\usepackage{chngcntr}
%\usepackage{enumerate}

%\usepackage{wasysym}
%\newcounter{MYtempeqncnt}
\DeclareMathOperator*{\Res}{Res}
%\renewcommand{\baselinestretch}{2}
\renewcommand\thesection{\arabic{section}}
\renewcommand\thesubsection{\thesection.\arabic{subsection}}
\renewcommand\thesubsubsection{\thesubsection.\arabic{subsubsection}}

\renewcommand\thesectiondis{\arabic{section}}
\renewcommand\thesubsectiondis{\thesectiondis.\arabic{subsection}}
\renewcommand\thesubsubsectiondis{\thesubsectiondis.\arabic{subsubsection}}

% correct bad hyphenation here
\hyphenation{op-tical net-works semi-conduc-tor}
\def\inputGnumericTable{}                                 %%

\lstset{
%language=C,
frame=single, 
breaklines=true,
columns=fullflexible
}
%\lstset{
%language=tex,
%frame=single, 
%breaklines=true
%}

\begin{document}
%


\newtheorem{theorem}{Theorem}[section]
\newtheorem{problem}{Problem}
\newtheorem{proposition}{Proposition}[section]
\newtheorem{lemma}{Lemma}[section]
\newtheorem{corollary}[theorem]{Corollary}
\newtheorem{example}{Example}[section]
\newtheorem{definition}[problem]{Definition}
%\newtheorem{thm}{Theorem}[section] 
%\newtheorem{defn}[thm]{Definition}
%\newtheorem{algorithm}{Algorithm}[section]
%\newtheorem{cor}{Corollary}
\newcommand{\parallelsum}{\mathbin{\|}}
\newcommand{\BEQA}{\begin{eqnarray}}
\newcommand{\EEQA}{\end{eqnarray}}
\newcommand{\define}{\stackrel{\triangle}{=}}
\bibliographystyle{IEEEtran}
%\bibliographystyle{ieeetr}
\providecommand{\mbf}{\mathbf}
\providecommand{\pr}[1]{\ensuremath{\Pr\left(#1\right)}}
\providecommand{\qfunc}[1]{\ensuremath{Q\left(#1\right)}}
\providecommand{\sbrak}[1]{\ensuremath{{}\left[#1\right]}}
\providecommand{\lsbrak}[1]{\ensuremath{{}\left[#1\right.}}
\providecommand{\rsbrak}[1]{\ensuremath{{}\left.#1\right]}}
\providecommand{\brak}[1]{\ensuremath{\left(#1\right)}}
\providecommand{\lbrak}[1]{\ensuremath{\left(#1\right.}}
\providecommand{\rbrak}[1]{\ensuremath{\left.#1\right)}}
\providecommand{\cbrak}[1]{\ensuremath{\left\{#1\right\}}}
\providecommand{\lcbrak}[1]{\ensuremath{\left\{#1\right.}}
\providecommand{\rcbrak}[1]{\ensuremath{\left.#1\right\}}}
\theoremstyle{remark}
\newtheorem{rem}{Remark}
\newcommand{\sgn}{\mathop{\mathrm{sgn}}}
\providecommand{\abs}[1]{\left\vert#1\right\vert}
\providecommand{\res}[1]{\Res\displaylimits_{#1}} 
\providecommand{\norm}[1]{\left\lVert#1\right\rVert}
%\providecommand{\norm}[1]{\lVert#1\rVert}
\providecommand{\mtx}[1]{\mathbf{#1}}
\providecommand{\mean}[1]{E\left[ #1 \right]}
\providecommand{\fourier}{\overset{\mathcal{F}}{ \rightleftharpoons}}
%\providecommand{\hilbert}{\overset{\mathcal{H}}{ \rightleftharpoons}}
\providecommand{\system}{\overset{\mathcal{H}}{ \longleftrightarrow}}
	%\newcommand{\solution}[2]{\textbf{Solution:}{#1}}
\newcommand{\solution}{\noindent \textbf{Solution: }}
\newcommand{\cosec}{\,\text{cosec}\,}
\providecommand{\dec}[2]{\ensuremath{\overset{#1}{\underset{#2}{\gtrless}}}}
\newcommand{\myvec}[1]{\ensuremath{\begin{pmatrix}#1\end{pmatrix}}}
\newcommand{\mydet}[1]{\ensuremath{\begin{vmatrix}#1\end{vmatrix}}}
%\numberwithin{equation}{section}
\numberwithin{equation}{subsection}
%\numberwithin{problem}{section}
%\numberwithin{definition}{section}
\makeatletter
\@addtoreset{figure}{problem}
\makeatother
\let\StandardTheFigure\thefigure
\let\vec\mathbf
%\renewcommand{\thefigure}{\theproblem.\arabic{figure}}
\renewcommand{\thefigure}{\theproblem}
%\setlist[enumerate,1]{before=\renewcommand\theequation{\theenumi.\arabic{equation}}
%\counterwithin{equation}{enumi}
%\renewcommand{\theequation}{\arabic{subsection}.\arabic{equation}}
\def\putbox#1#2#3{\makebox[0in][l]{\makebox[#1][l]{}\raisebox{\baselineskip}[0in][0in]{\raisebox{#2}[0in][0in]{#3}}}}
     \def\rightbox#1{\makebox[0in][r]{#1}}
     \def\centbox#1{\makebox[0in]{#1}}
     \def\topbox#1{\raisebox{-\baselineskip}[0in][0in]{#1}}
     \def\midbox#1{\raisebox{-0.5\baselineskip}[0in][0in]{#1}}
\vspace{3cm}
\title{
%	\logo{
Feedback Voltage Amplifier: Shunt-Shunt
%	}
}
\author{ Deep$^{*}$% <-this % stops a space
	\thanks{*The author is with the Department
		of Electrical Engineering, Indian Institute of Technology, Hyderabad
		502285 India. All content in this manual is released under GNU GPL.  Free and open source.}
	
}	
%\title{
%	\logo{Matrix Analysis through Octave}{\begin{center}\includegraphics[scale=.24]{tlc}\end{center}}{}{HAMDSP}
%}
% paper title
% can use linebreaks \\ within to get better formatting as desired
%\title{Matrix Analysis through Octave}
%
%
% author names and IEEE memberships
% note positions of commas and nonbreaking spaces ( ~ ) LaTeX will not break
% a structure at a ~ so this keeps an author's name from being broken across
% two lines.
% use \thanks{} to gain access to the first footnote area
% a separate \thanks must be used for each paragraph as LaTeX2e's \thanks
% was not built to handle multiple paragraphs
%
%\author{<-this % stops a space
%\thanks{}}
%}
% note the % following the last \IEEEmembership and also \thanks - 
% these prevent an unwanted space from occurring between the last author name
% and the end of the author line. i.e., if you had this:
% 
% \author{....lastname \thanks{...} \thanks{...} }
%                     ^------------^------------^----Do not want these spaces!
%
% a space would be appended to the last name and could cause every name on that
% line to be shifted left slightly. This is one of those "LaTeX things". For
% instance, "\textbf{A} \textbf{B}" will typeset as "A B" not "AB". To get
% "AB" then you have to do: "\textbf{A}\textbf{B}"
% \thanks is no different in this regard, so shield the last } of each \thanks
% that ends a line with a % and do not let a space in before the next \thanks.
% Spaces after \IEEEmembership other than the last one are OK (and needed) as
% you are supposed to have spaces between the names. For what it is worth,
% this is a minor point as most people would not even notice if the said evil
% space somehow managed to creep in.
% The paper headers
%\markboth{Journal of \LaTeX\ Class Files,~Vol.~6, No.~1, January~2007}%
%{Shell \MakeLowercase{\textit{et al.}}: Bare Demo of IEEEtran.cls for Journals}
% The only time the second header will appear is for the odd numbered pages
% after the title page when using the twoside option.
% 
% *** Note that you probably will NOT want to include the author's ***
% *** name in the headers of peer review papers.                   ***
% You can use \ifCLASSOPTIONpeerreview for conditional compilation here if
% you desire.
% If you want to put a publisher's ID mark on the page you can do it like
% this:
%\IEEEpubid{0000--0000/00\$00.00~\copyright~2007 IEEE}
% Remember, if you use this you must call \IEEEpubidadjcol in the second
% column for its text to clear the IEEEpubid mark.
% make the title area
\maketitle
\newpage
\tableofcontents
\bigskip
\renewcommand{\thefigure}{\theenumi}
\renewcommand{\thetable}{\theenumi}
%\renewcommand{\theequation}{\theenumi}
%\begin{abstract}
%%\boldmath
%In this letter, an algorithm for evaluating the exact analytical bit error rate  (BER)  for the piecewise linear (PL) combiner for  multiple relays is presented. Previous results were available only for upto three relays. The algorithm is unique in the sense that  the actual mathematical expressions, that are prohibitively large, need not be explicitly obtained. The diversity gain due to multiple relays is shown through plots of the analytical BER, well supported by simulations. 
%
%\end{abstract}
% IEEEtran.cls defaults to using nonbold math in the Abstract.
% This preserves the distinction between vectors and scalars. However,
% if the journal you are submitting to favors bold math in the abstract,
% then you can use LaTeX's standard command \boldmath at the very start
% of the abstract to achieve this. Many IEEE journals frown on math
% in the abstract anyway.
% Note that keywords are not normally used for peerreview papers.
%\begin{IEEEkeywords}
%Cooperative diversity, decode and forward, piecewise linear
%\end{IEEEkeywords}
% For peer review papers, you can put extra information on the cover
% page as needed:
% \ifCLASSOPTIONpeerreview
% \begin{center} \bfseries EDICS Category: 3-BBND \end{center}
% \fi
%
% For peerreview papers, this IEEEtran command inserts a page break and
% creates the second title. It will be ignored for other modes.
%\IEEEpeerreviewmaketitle
%\begin{abstract}
%This manual is an introduction to control systems based on GATE problems.Links to sample Python codes are available in the text.  
%\end{abstract}
%Download python codes using 
%\begin{lstlisting}
%svn co https://github.com/gadepall/school/trunk/control/codes
%\end{lstlisting}
%\section{Feedback Voltage Amplifier: Shunt-Shunt}
%%%%%%%%%%%%%%%%%%%%%%%%%%%%%%%%%%%%%%%%%%%%%%%%%%%%%%%%%%%%%%%%%%%%%%
%%                                                                  %%
%%  This is the header of a LaTeX2e file exported from Gnumeric.    %%
%%                                                                  %%
%%  This file can be compiled as it stands or included in another   %%
%%  LaTeX document. The table is based on the longtable package so  %%
%%  the longtable options (headers, footers...) can be set in the   %%
%%  preamble section below (see PRAMBLE).                           %%
%%                                                                  %%
%%  To include the file in another, the following two lines must be %%
%%  in the including file:                                          %%
%%        \def\inputGnumericTable{}                                 %%
%%  at the beginning of the file and:                               %%
%%        \input{name-of-this-file.tex}                             %%
%%  where the table is to be placed. Note also that the including   %%
%%  file must use the following packages for the table to be        %%
%%  rendered correctly:                                             %%
%%    \usepackage[latin1]{inputenc}                                 %%
%%    \usepackage{color}                                            %%
%%    \usepackage{array}                                            %%
%%    \usepackage{longtable}                                        %%
%%    \usepackage{calc}                                             %%
%%    \usepackage{multirow}                                         %%
%%    \usepackage{hhline}                                           %%
%%    \usepackage{ifthen}                                           %%
%%  optionally (for landscape tables embedded in another document): %%
%%    \usepackage{lscape}                                           %%
%%                                                                  %%
%%%%%%%%%%%%%%%%%%%%%%%%%%%%%%%%%%%%%%%%%%%%%%%%%%%%%%%%%%%%%%%%%%%%%%



%%  This section checks if we are begin input into another file or  %%
%%  the file will be compiled alone. First use a macro taken from   %%
%%  the TeXbook ex 7.7 (suggestion of Han-Wen Nienhuys).            %%
\def\ifundefined#1{\expandafter\ifx\csname#1\endcsname\relax}


%%  Check for the \def token for inputed files. If it is not        %%
%%  defined, the file will be processed as a standalone and the     %%
%%  preamble will be used.                                          %%
\ifundefined{inputGnumericTable}

%%  We must be able to close or not the document at the end.        %%
	\def\gnumericTableEnd{\end{document}}


%%%%%%%%%%%%%%%%%%%%%%%%%%%%%%%%%%%%%%%%%%%%%%%%%%%%%%%%%%%%%%%%%%%%%%
%%                                                                  %%
%%  This is the PREAMBLE. Change these values to get the right      %%
%%  paper size and other niceties.                                  %%
%%                                                                  %%
%%%%%%%%%%%%%%%%%%%%%%%%%%%%%%%%%%%%%%%%%%%%%%%%%%%%%%%%%%%%%%%%%%%%%%

	\documentclass[12pt%
			  %,landscape%
                    ]{report}
       \usepackage[latin1]{inputenc}
       \usepackage{fullpage}
       \usepackage{color}
       \usepackage{array}
       \usepackage{longtable}
       \usepackage{calc}
       \usepackage{multirow}
       \usepackage{hhline}
       \usepackage{ifthen}

	\begin{document}


%%  End of the preamble for the standalone. The next section is for %%
%%  documents which are included into other LaTeX2e files.          %%
\else

%%  We are not a stand alone document. For a regular table, we will %%
%%  have no preamble and only define the closing to mean nothing.   %%
    \def\gnumericTableEnd{}

%%  If we want landscape mode in an embedded document, comment out  %%
%%  the line above and uncomment the two below. The table will      %%
%%  begin on a new page and run in landscape mode.                  %%
%       \def\gnumericTableEnd{\end{landscape}}
%       \begin{landscape}


%%  End of the else clause for this file being \input.              %%
\fi

%%%%%%%%%%%%%%%%%%%%%%%%%%%%%%%%%%%%%%%%%%%%%%%%%%%%%%%%%%%%%%%%%%%%%%
%%                                                                  %%
%%  The rest is the gnumeric table, except for the closing          %%
%%  statement. Changes below will alter the table's appearance.     %%
%%                                                                  %%
%%%%%%%%%%%%%%%%%%%%%%%%%%%%%%%%%%%%%%%%%%%%%%%%%%%%%%%%%%%%%%%%%%%%%%

\providecommand{\gnumericmathit}[1]{#1} 
%%  Uncomment the next line if you would like your numbers to be in %%
%%  italics if they are italizised in the gnumeric table.           %%
%\renewcommand{\gnumericmathit}[1]{\mathit{#1}}
\providecommand{\gnumericPB}[1]%
{\let\gnumericTemp=\\#1\let\\=\gnumericTemp\hspace{0pt}}
 \ifundefined{gnumericTableWidthDefined}
        \newlength{\gnumericTableWidth}
        \newlength{\gnumericTableWidthComplete}
        \newlength{\gnumericMultiRowLength}
        \global\def\gnumericTableWidthDefined{}
 \fi
%% The following setting protects this code from babel shorthands.  %%
 \ifthenelse{\isundefined{\languageshorthands}}{}{\languageshorthands{english}}
%%  The default table format retains the relative column widths of  %%
%%  gnumeric. They can easily be changed to c, r or l. In that case %%
%%  you may want to comment out the next line and uncomment the one %%
%%  thereafter                                                      %%
\providecommand\gnumbox{\makebox[0pt]}
%%\providecommand\gnumbox[1][]{\makebox}

%% to adjust positions in multirow situations                       %%
\setlength{\bigstrutjot}{\jot}
\setlength{\extrarowheight}{\doublerulesep}

%%  The \setlongtables command keeps column widths the same across  %%
%%  pages. Simply comment out next line for varying column widths.  %%
\setlongtables

\setlength\gnumericTableWidth{%
	70pt+%
	70pt+%
0pt}
\def\gumericNumCols{3}
\setlength\gnumericTableWidthComplete{\gnumericTableWidth+%
         \tabcolsep*\gumericNumCols*2+\arrayrulewidth*\gumericNumCols}
\ifthenelse{\lengthtest{\gnumericTableWidthComplete > \linewidth}}%
         {\def\gnumericScale{\ratio{\linewidth-%
                        \tabcolsep*\gumericNumCols*2-%
                        \arrayrulewidth*\gumericNumCols}%
{\gnumericTableWidth}}}%
{\def\gnumericScale{1}}

%%%%%%%%%%%%%%%%%%%%%%%%%%%%%%%%%%%%%%%%%%%%%%%%%%%%%%%%%%%%%%%%%%%%%%
%%                                                                  %%
%% The following are the widths of the various columns. We are      %%
%% defining them here because then they are easier to change.       %%
%% Depending on the cell formats we may use them more than once.    %%
%%                                                                  %%
%%%%%%%%%%%%%%%%%%%%%%%%%%%%%%%%%%%%%%%%%%%%%%%%%%%%%%%%%%%%%%%%%%%%%%

\ifthenelse{\isundefined{\gnumericColA}}{\newlength{\gnumericColA}}{}\settowidth{\gnumericColA}{\begin{tabular}{@{}p{70pt*\gnumericScale}@{}}x\end{tabular}}
\ifthenelse{\isundefined{\gnumericColB}}{\newlength{\gnumericColB}}{}\settowidth{\gnumericColB}{\begin{tabular}{@{}p{70pt*\gnumericScale}@{}}x\end{tabular}}

\begin{tabular}[c]{%
	b{\gnumericColA}%
	b{\gnumericColB}%
	}

%%%%%%%%%%%%%%%%%%%%%%%%%%%%%%%%%%%%%%%%%%%%%%%%%%%%%%%%%%%%%%%%%%%%%%
%%  The longtable options. (Caption, headers... see Goosens, p.124) %%
%	\caption{The Table Caption.}             \\	%
% \hline	% Across the top of the table.
%%  The rest of these options are table rows which are placed on    %%
%%  the first, last or every page. Use \multicolumn if you want.    %%

%%  Header for the first page.                                      %%
%	\multicolumn{3}{c}{The First Header} \\ \hline 
%	\multicolumn{1}{c}{colTag}	%Column 1
%	&\multicolumn{1}{c}{colTag}	%Column 2
%	&\multicolumn{1}{c}{colTag}	\\ \hline %Last column
%	\endfirsthead

%%  The running header definition.                                  %%
%	\hline
%	\multicolumn{3}{l}{\ldots\small\slshape continued} \\ \hline
%	\multicolumn{1}{c}{colTag}	%Column 1
%	&\multicolumn{1}{c}{colTag}	%Column 2
%	&\multicolumn{1}{c}{colTag}	\\ \hline %Last column
%	\endhead

%%  The running footer definition.                                  %%
%	\hline
%	\multicolumn{3}{r}{\small\slshape continued\ldots} \\
%	\endfoot

%%  The ending footer definition.                                   %%
%	\multicolumn{3}{c}{That's all folks} \\ \hline 
%	\endlastfoot
%%%%%%%%%%%%%%%%%%%%%%%%%%%%%%%%%%%%%%%%%%%%%%%%%%%%%%%%%%%%%%%%%%%%%%

\hhline{|-|-|-}
	 \multicolumn{1}{|p{\gnumericColA}|}%
	{\gnumericPB{\centering}\textbf{Damping ratio($\zeta$)}}
	&\multicolumn{1}{p{\gnumericColB}|}%
	{\gnumericPB{\centering}\textbf{Undamped natural frequency($\omega_{n}$)}}
\\
\hhline{|---|}
	 \multicolumn{1}{|p{\gnumericColA}|}%
	{\gnumericPB{\raggedright}Damping ratio basically indicates the amount of damping present in the overall system denoted by zeta, where damping is a counter force.It is a dimensionless measure describing how oscillations in a system decay after a disturbance.}
	&\multicolumn{1}{p{\gnumericColB}|}%
	{\gnumericPB{\raggedright}The frequency of oscillation of the system without damping.A system may or may not have an associated natural frequency.}
\\
\hhline{|---|}
	 \multicolumn{1}{|p{\gnumericColA}|}%
	{\gnumericPB{\centering}The damping ratio is a system parameter, denoted by $\zeta$, that can vary from undamped ($\zeta$ = 0), underdamped ($\zeta$ \textless 1) through critically damped ($\zeta$ = 1) to overdamped ($\zeta$ \textgreater 1).}
	&\multicolumn{1}{p{\gnumericColB}|}%
	{\gnumericPB{\raggedright}Only systems with $\zeta$ \textless 1 have a natural frequency $\omega$ and only in the case that $\zeta$ = 0 will the natural frequency $\omega$ = $\omega_{n}$, the undamped natural frequency.}
\\
\hhline{|-|-|-|}
\end{tabular}

\ifthenelse{\isundefined{\languageshorthands}}{}{\languageshorthands{\languagename}}
\gnumericTableEnd

%\section{Stabilty}
%%\begin{enumerate}[label=\arabic*.,ref=\theenumi]
\begin{enumerate}[label=\thesection.\arabic*.,ref=\thesection.\theenumi]
\numberwithin{equation}{enumi}
%3.1
\item Discuss the relation between Stability and Phase Margin (PM).\\
\solution Let the loop gain
\begin{align}
L(s) = G(s)H(s)
\end{align}
Then the closed loop gain 
\begin{align}
T\brak{\j\omega} = \frac{G\brak{\j\omega}}{1+L\brak{\j\omega}}%\\
\end{align}
and 
%
If 
\begin{align}
\abs{L\brak{\j\omega_0}} &= 1,
\\
T\brak{\j\omega_0} &= \frac{G\brak{\j\omega_0}}{1+L\brak{\j\omega_0}}\\
\\
 &= \frac{G\brak{\j\omega_0}}{1+\exp\cbrak{\phase{L\brak{\j\omega_0}}}},
\end{align}
and 
\begin{align}
PM = 180^{\degree} - \phase{L\brak{\j\omega_0}}
\end{align}
%
and for
\begin{align}
PM
\begin{cases}
> 0 & T(s) \text{ stable}
\\
= 0 & T(s) \text{ marginally stable}
\\
< 0 & T(s) \text{ unstable}
\end{cases}
\end{align}
%------------------------------------------------------------------------------------------------------%
% 3.2
%\item If Phase of Loop Gain is $\phi$ at which the magnitude of loop gain is unity. Find range of $\phi$ for the system to be Stable, Marginally Stable and Unstable.\\
%\solution
%\begin{align}
%\phi = \alpha - 180^{\degree}\\
%\alpha = \phi + 180^{\degree}
%\end{align}
%
%For Stable,
%\begin{align}
%\alpha > 0\\
%\phi > -180^{\degree}
%\end{align}
%
%For Marginally Stable,
%\begin{align}
%\alpha = 0\\
%\phi = -180^{\degree}
%\end{align}
%
%For Unstable,
%\begin{align}
%\alpha < 0\\
%\phi < -180^{\degree}
%\end{align}
%------------------------------------------------------------------------------------------------------%
%3.3
\item For constant $H$, find the frequency at which $\phase{L\brak{\j\omega}} = -180^{\degree}$ and determine the region for Stability.
\solution

\begin{align}
\phase{G(f)H(f)} = \phase{G(f)}
\end{align}
\begin{multline}
\implies \phase{G(f)} = -180\degree
\\
=-\sbrak{\tan ^{-1}\brak{\frac{f}{10^{5}}}+\tan ^{-1}\brak{\frac{f}{10^{6}}}+\tan ^{-1}\brak{\frac{f}{10^{7}}}}
\end{multline}
or,
\begin{align}
f = f_{\pi} = 3.34 M Hz.
\end{align}

So, for 
\begin{itemize}
\item $f > 3.34 M Hz$, System is Unstable
\item $f = 3.34 M Hz$, System is Marginally Stable
\item $f < 3.34 M Hz$, System is Stable
\end{itemize}
%------------------------------------------------------------------------------------------------------%
%3.4
\item Determine the range of $H$ for Stability.\\
\solution
\begin{align}
\abs{G(f_{\pi})} &= 320 - 40\log(f_{\pi})\\
 &= 59 dB = 896
H &= 1.11 \times 10^{-3} \brak{\because \abs{G(f_{\pi})H} = 1}
\end{align}

%If $f$ \textbf{decreases} below $f = 3.34\times 10^6 Hz$, $G$ increases and the value of $H$ at which Loop-Gain becomes unity \textbf{decreases} below $H = 1.11 \times 10^{-3}$.

Thus, 
\begin{itemize}
\item $H > 1.11 \times 10^{-3}$, System is Unstable
\item $H = 1.11 \times 10^{-3}$, System is Marginally Stable
\item $H < 1.11 \times 10^{-3}$, System is Stable
\end{itemize}

\item Verify the stablity from the value of $H$\\
\solution
Run the following code to verify the results
\begin{lstlisting}
codes/ee18btech11014/Stability.py
\end{lstlisting}
\end{enumerate}

%\section{Phase Margin}
%%\begin{enumerate}[label=\arabic*.,ref=\theenumi]
\begin{enumerate}[label=\thesection.\arabic*.,ref=\thesection.\theenumi]
\numberwithin{equation}{enumi}
%-------------------------------------------------------------------------------------------------%
%2.1.6
\item Find the frequency for which $PM = 90 \degree$.  Assume $H$ to be constant.
\\
%\solution Letting 
%\item Find the frequencies for which phase margins are $90\degree$ and $45\degree$ respectively?\\
\solution $\because \phase{H\brak{f}} = 1$, 
\begin{align}
\phase{G\brak{f_{90}}H\brak{f_{90}}} &= \phase{G\brak{f_{90}}} = 90\degree - 180\degree
\\
&= -90\degree
\label{eq:ee18btech11014_Gpm90}
%\\
%\implies \abs{G\brak{f_{90}}H\brak{f_{90}}} &=1
\end{align}
%
%From \eqref{eq:ee18btech11014_G_ang},
%%
%\begin{multline}
%\phi\brak{f} =
%\\
%-\sbrak{\tan ^{-1}\brak{\frac{f}{10^{5}}}+\tan ^{-1}\brak{\frac{f}{10^{6}}}+\tan ^{-1}\brak{\frac{f}{10^{7}}}}
%\end{multline}
The Bode plot in Fig. 	\ref{fig:ee18btech11014_Bode} shows that 
\begin{align}
\abs{G(f)} < 1, \quad f > 10^8
\end{align}
%
Also, 
\begin{align}
\tan^{-1}\brak{\frac{f}{10^{7}}} \approx 0, \quad f < 10^8
\end{align}

Thus, from  \eqref{eq:ee18btech11014_G_ang} and \eqref{eq:ee18btech11014_Gpm90},
%
\begin{align}
\phi\brak{f} &\approx
-\sbrak{\tan ^{-1}\brak{\frac{f}{10^{5}}}+\tan ^{-1}\brak{\frac{f}{10^{6}}}}
\\
&= -90 \degree
\\
\implies f_{90} &= 3.162 \times 10^{5}
\end{align}
after simplification.
%-------------------------------------------------------------------------------------------------%
\item Find $H$ when the $PM = 90 \degree$.
\\
\solution By definition of the PM, 
\begin{align}
\abs{G\brak{f_{90}}H\brak{f_{90}}} &=1
\\
\implies \abs{H\brak{f_{90}}} &=\frac{1}{\abs{G\brak{f_{90}}}}
\label{eq:ee18btech11014_GH_PM_90}
\end{align}
%
From \eqref{eq:ee18btech11014_G_piece},
\begin{align}
20 \log \abs{G(f)} &= 200 - 20\log(3.162 \times 10^{5})\\
&= 90 dB \\
\implies \abs{G(f)} &= 3.1625 \times 10^{4}
\\
\implies H &= 3.162 \times 10^{-5}
\end{align}
using \eqref{eq:ee18btech11014_GH_PM_90}.
%-------------------------------------------------------------------------------------------------%
\item Design the closed loop circuit for $PM = 90 \degree$
\\
\solution See Fig. 	\ref{fig:ee18btech11014_Closed-Loop Circuit alpha=90}, where Fig. 	\ref{fig:ee18btech11014_Feedback Circuit} is used for the feedback $H$ with $R_M = 0.3162 M \ohm$ and 	$R_F = 10 \ohm$.

\begin{figure}[ht!]
	\begin{center}
		\resizebox{\columnwidth}{!}{\begin{circuitikz}[american]

\draw (2,2)  node[op amp] (OA) {};
\draw (OA.up) -- ++(0, 0.3) node[vcc]{$+10V$};
\draw (OA.down) -- ++(0,-0.3) node[vee]{$-10V$};
\draw (OA.+) -- (0,1.5) to[vsourcesin, l= $v_{s}$] (0,0) node[ground](GND){};
\draw (OA.-) -- (0,2.5) to[R=$10\ohm$] (-2,2.5) node[ground, rotate=270](GND){};
\draw (OA.out) -- (3,2) node[label={below:$v_{a}$}]{};
\draw (3,2) to[R=$10^{2}\ohm$] (5.5,2) node[label={above:$v_{b}$}]{} to[C,l_=$\frac{10^{-9}}{2\pi}F$] (5.5,0) node[ground](GND){};
\draw (5.5,2) to[R=$10^{3}\ohm$] (8,2) node[label={above:$v_{c}$}]{} to[C,l_=$\frac{10^{-9}}{2\pi}F$] (8,0) node[ground](GND){};
\draw (8,2) to[R=$10^{4}\ohm$] (10.5,2) to[C,l_=$\frac{10^{-9}}{2\pi}F$] (10.5,0) node[ground](GND){};
\draw (10.5,2) -- (11.5,2) node[label={above:$v_{o}$}]{};
\draw (10.5,2) -- (10.5,4) to[R=$3.162\times 10^{5}\ohm$] (0,4) -- (0,2.5);

\end{circuitikz}
}
	\end{center}
	\caption{}
	\label{fig:ee18btech11014_Closed-Loop Circuit alpha=90}
\end{figure}

%-------------------------------------------------------------------------------------------------%
\item Repeat all the above for $PM = 45\degree$.
%-------------------------------------------------------------------------------------------------%

%-\tan^{-1}\left(f/10^{5}\right)-\tan^{-1}\left(f/10^{6}\right) = -90\\
%\tan^{-1}\left(f/10^{5}\right)+\tan^{-1}\left(f/10^{6}\right) = 90\\
%\tan^{-1}\left(f/10^{5}\right) = 90-\tan^{-1}\left(f/10^{6}\right)\\
%\tan^{-1}\left(f/10^{5}\right) = \cot^{-1}\left(f/10^{6}\right)\\
%\tan^{-1}\left(f/10^{5}\right) = \tan^{-1}\left(10^{6}/f\right)\\
%f^{2} = 10^{11}\\
%f = 3.162 \times 10^{5}
%\end{align}
%
%So, the approximate value of $f$ at which Phase Margin is $90\degree$ is $f=3.162 \times 10^{5} Hz$.\\

%Similarly let Phase Margin be $\alpha = 45\degree$. Then,
%\begin{align}
%\alpha = \phi - (-180\degree)\\
%\phi = -180\degree + \alpha\\
%\phi = -135\degree
%\end{align}
%
%So, by the definition of Phase-Margin, at $\phi = -135\degree$ , $|GH| = 1 $.  The value of $\phi = -135\degree$ aproximately at poles $f=10^{6} Hz$. 
%
%So, the approximate value of $f$ at which Phase Margin is $45\degree$ is $f=10^{6}$.\\
%%-------------------------------------------------------------------------------------------------%
%%2.1.7
%\item Find the minimum values of Closed-Loop Voltage Gain for which phase margins are $90\degree$ and  $45\degree$ respectively\\
%\solution\\
%For $\alpha=90\degree$,
%\begin{align}
%f=3.162 \times 10^{5}
%\end{align}
%By substituting $f$ in Open-Loop Gain $G(f)$ (assuming poles are far part), 
%\begin{align}
%G(f) = 200 - 20log(3.162 \times 10^{5})\\
%G(f) = 90 dB \\
%G = 3.1625 \times 10^{4}
%\end{align}
%
%At that $f=3.162 \times 10^{5}$, 
%\begin{align}
%H = \frac{1}{G}\\
%H = 3.162 \times 10^{-5}
%\end{align}
%
%The minimum value of Closed-Loop Gain occurs at $|GH| \gg 1$ and the value of Closed-Loop Gain is $T=\frac{1}{H}$
%
%\begin{align}
%T = \frac{1}{H} = 3.1625 \times 10^{4}
%\end{align}
%
%\textbf{So, The minimum value of Closed-Loop Gain with Phase Margin equal to $\alpha=90\degree$ is $T_{min} = 3.1625 \times 10^{4}$.}\\
%
%For $\alpha=45\degree$,
%\begin{align}
%f=10^{6}
%\end{align}
%By substituting $f$ in Open-Loop Gain $G(f)$ (assuming poles are far part), 
%\begin{align}
%G(f) = 200 - 20log(10^{6})\\
%G(f) = 80 dB \\
%G = 10^{4}
%\end{align}
%
%At that $f = 10^{6}$, 
%\begin{align}
%H = \frac{1}{G}\\
%H = 10^{-4}
%\end{align}
%
%The minimum value of Closed-Loop Gain occurs at $|GH| \gg 1$ and the value of Closed-Loop Gain is $T=\frac{1}{H}$
%
%\begin{align}
%T = \frac{1}{H} = 10^{4}
%\end{align}
%
%\textbf{So, The minimum value of Closed-Loop Gain with Phase Margin equal to $\alpha=45\degree$ is $T_{min} = 10^{4}$.}\\
%%-------------------------------------------------------------------------------------------------%
%
%\item Design a Feedback circuit for Phase Margin $\alpha=45^{\circ}$.\\
%\solution
%\begin{figure}[ht!]
%	\begin{center}
%		\resizebox{\columnwidth/2}{!}{\begin{circuitikz}[american]
\ctikzset{tripoles/mos style/arrows}
\draw (1,2) to[short, -o] (0,2) node[label={below:$v_{o}$}]{};
\draw (1,2) to[R=$100k\ohm$] (2,2) -- (3,2) to[R=$10\ohm$] (3,0) node[ground](GND){};
\draw (3,2) to[short, -o] (4,2) node[label={below:$v_{f}$}]{};
\end{circuitikz}
}
%	\end{center},
%	\caption{}
%	\label{fig:ee18btech11014_alpha=45}
%\end{figure}
%\begin{align}
%v_{f} = \frac{10}{10 + 10^{5}} \times v_{o}\\
%v_{f} \approx 10^{-4} v_{o}\\
%\frac{v_{f}}{v_{o}} \approx 10^{-4}\\
%H(s) = 10^{-4}
%\end{align}
%%-------------------------------------------------------------------------------------------------%
%
%\item Design a Feedback circuit for Phase Margin $\alpha=90^{\circ}$.\\
%\solution
%\begin{figure}[ht!]
%	\begin{center}
%		\resizebox{\columnwidth/2}{!}{\begin{circuitikz}[american]
\ctikzset{tripoles/mos style/arrows}
\draw (1,2) to[short, -o] (0,2) node[label={below:$v_{o}$}]{};
\draw (1,2) to[R=$0.3162 M\ohm$] (2,2) -- (3,2) to[R=$10\ohm$] (3,0) node[ground](GND){};
\draw (3,2) to[short, -o] (4,2) node[label={below:$v_{f}$}]{};
\end{circuitikz}
}
%	\end{center},
%	\caption{}
%	\label{fig:ee18btech11014_alpha=90}
%\end{figure}
%\begin{align}
%v_{f} = \frac{10}{10 + 3.162\times 10^{5}} \times v_{o}\\
%v_{f} \approx 3.162\times 10^{-5} v_{o}\\
%\frac{v_{f}}{v_{o}} \approx 3.162\times 10^{-5}\\
%H(s) = 3.162\times 10^{-5}
%\end{align}
%%-------------------------------------------------------------------------------------------------%
%
%\item  Design a Closed-Loop Transfer Function by combining both the Open-Loop and Feedback Circuits for phase Margin $\alpha=45^{\circ}$. Also draw its Equivalent Circuit\\
%\solution\\
%The Closed-Loop Circuit is
%\begin{figure}[ht!]
%	\begin{center}
%		\resizebox{\columnwidth}{!}{\begin{circuitikz}[american]

\draw (2,2)  node[op amp] (OA) {};
\draw (OA.up) -- ++(0, 0.3) node[vcc]{$+10V$};
\draw (OA.down) -- ++(0,-0.3) node[vee]{$-10V$};
\draw (OA.+) -- (0,1.5) to[vsourcesin, l= $v_{s}$] (0,0) node[ground](GND){};
\draw (OA.-) -- (0,2.5) to[R=$10\ohm$] (-2,2.5) node[ground, rotate=270](GND){};
\draw (OA.out) -- (3,2) node[label={below:$v_{a}$}]{};
\draw (3,2) to[R=$10^{2}\ohm$] (5.5,2) node[label={above:$v_{b}$}]{} to[C,l_=$\frac{10^{-9}}{2\pi}F$] (5.5,0) node[ground](GND){};
\draw (5.5,2) to[R=$10^{3}\ohm$] (8,2) node[label={above:$v_{c}$}]{} to[C,l_=$\frac{10^{-9}}{2\pi}F$] (8,0) node[ground](GND){};
\draw (8,2) to[R=$10^{4}\ohm$] (10.5,2) to[C,l_=$\frac{10^{-9}}{2\pi}F$] (10.5,0) node[ground](GND){};
\draw (10.5,2) -- (11.5,2) node[label={above:$v_{o}$}]{};
\draw (10.5,2) -- (10.5,4) to[R=$10^{5}\ohm$] (0,4) -- (0,2.5);

\end{circuitikz}
}
%	\end{center}
%	\caption{}
%	\label{fig:ee18btech11014_Closed-Loop Circuit alpha=45}
%\end{figure}
%
%The Equivalent Circuit of Closed-Loop Circuit is
%\begin{figure}[ht!]
%	\begin{center}
%		\resizebox{\columnwidth}{!}{\begin{circuitikz}[american]-1
\draw (-3,0) node[ground](GND){} to[vsourcesin, l= $v_{s}$] (-3,2) to[short,-o] (0.25,2) node[label={below:$+$}]{};
\draw (0,0.1) to[R=$10\ohm$,v=$v_{f}$] (-2,0.1) node[ground](GND){}; 
\draw (0,0.1) -- (1,0.1) -- (1,4) to[R=$10^{5}\ohm$] (10.5,4) -- (10.5,2);


\draw (0.25,0.1) to[short,-o] (0.25,0.1) node[label={above:$-$}]{};
\draw (0.25,0.625) node[label={$v_{i}$}] {};


\draw (3,2) node[label={above:$v_{a}$}]{};
\draw (3,0) node[ground](GND){} to[vsourcesin, l= $10^5 v_{i}$] (3,2);
\draw (3,2) to[R=$10^{2}\ohm$] (5.5,2) node[label={above:$v_{b}$}]{} to[C,l_=$\frac{10^{-9}}{2\pi}F$] (5.5,0) node[ground](GND){};
\draw (5.5,2) to[R=$10^{3}\ohm$] (8,2) node[label={above:$v_{c}$}]{} to[C,l_=$\frac{10^{-9}}{2\pi}F$] (8,0) node[ground](GND){};
\draw (8,2) to[R=$10^{4}\ohm$] (10.5,2) to[C,l_=$\frac{10^{-9}}{2\pi}F$] (10.5,0) node[ground](GND){};
\draw (10.5,2) -- (11.5,2) node[label={above:$v_{o}$}]{};

\end{circuitikz}
}
%	\end{center},
%	\caption{}
%	\label{fig:ee18btech11014_Closed-Loop Equivalent Circuit alpha=45}
%\end{figure}
%
%From the Equivalent Circuit Diagram,
%\begin{align}
%G(s) = \dfrac{10^5}{\left(1+\frac{s}{2\pi 10^{5}}\right)\left(1+\frac{s}{2\pi 10^{6}}\right)\left(1+\frac{s}{2\pi 10^{7}}\right)}\\
%H(s) = \frac{v_{f}}{v_{o}} = 10^{-4}
%\end{align}
%
%The Closed-Loop Gain,
%\begin{align}
%v_{i} = v_{s} - v_{f}\\
%\frac{v_{o}}{G} = v_{s} - Hv_{o}\\
%\frac{v_{o}}{v_{s}} = \frac{G}{1+GH}
%\end{align}
%
%So, the Closed-Loop Gain,
%\begin{align}
%T(s) = \frac{v_{o}}{v_{s}} = \dfrac{10^5}{10 + \left(1+s\frac{s}{2\pi 10^{5}}\right)\left(1+\frac{s}{2\pi 10^{6}}\right)\left(1+j\frac{s}{2\pi 10^{7}}\right)}
%\end{align}
%%-------------------------------------------------------------------------------------------------%
%
%\item  Design a Closed-Loop Transfer Function by combining both the Open-Loop and Feedback Circuits for phase Margin $\alpha=90^{\circ}$. Also draw its Equivalent Circuit\\
%\solution\\
%The Closed-Loop Circuit is
%\begin{figure}[ht!]
%	\begin{center}
%		\resizebox{\columnwidth}{!}{\begin{circuitikz}[american]

\draw (2,2)  node[op amp] (OA) {};
\draw (OA.up) -- ++(0, 0.3) node[vcc]{$+10V$};
\draw (OA.down) -- ++(0,-0.3) node[vee]{$-10V$};
\draw (OA.+) -- (0,1.5) to[vsourcesin, l= $v_{s}$] (0,0) node[ground](GND){};
\draw (OA.-) -- (0,2.5) to[R=$10\ohm$] (-2,2.5) node[ground, rotate=270](GND){};
\draw (OA.out) -- (3,2) node[label={below:$v_{a}$}]{};
\draw (3,2) to[R=$10^{2}\ohm$] (5.5,2) node[label={above:$v_{b}$}]{} to[C,l_=$\frac{10^{-9}}{2\pi}F$] (5.5,0) node[ground](GND){};
\draw (5.5,2) to[R=$10^{3}\ohm$] (8,2) node[label={above:$v_{c}$}]{} to[C,l_=$\frac{10^{-9}}{2\pi}F$] (8,0) node[ground](GND){};
\draw (8,2) to[R=$10^{4}\ohm$] (10.5,2) to[C,l_=$\frac{10^{-9}}{2\pi}F$] (10.5,0) node[ground](GND){};
\draw (10.5,2) -- (11.5,2) node[label={above:$v_{o}$}]{};
\draw (10.5,2) -- (10.5,4) to[R=$3.162\times 10^{5}\ohm$] (0,4) -- (0,2.5);

\end{circuitikz}
}
%	\end{center}
%	\caption{}
%	\label{fig:ee18btech11014_Closed-Loop Circuit alpha=90}
%\end{figure}
%
%The Equivalent Circuit of Closed-Loop Circuit is
%\begin{figure}[ht!]
%	\begin{center}
%		\resizebox{\columnwidth}{!}{\begin{circuitikz}[american]-1
\draw (-3,0) node[ground](GND){} to[vsourcesin, l= $v_{s}$] (-3,2) to[short,-o] (0.25,2) node[label={below:$+$}]{};
\draw (0,0.1) to[R=$10\ohm$,v=$v_{f}$] (-2,0.1) node[ground](GND){}; 
\draw (0,0.1) -- (1,0.1) -- (1,4) to[R=$3.162\times 10^{5}\ohm$] (10.5,4) -- (10.5,2);


\draw (0.25,0.1) to[short,-o] (0.25,0.1) node[label={above:$-$}]{};
\draw (0.25,0.625) node[label={$v_{i}$}] {};


\draw (3,2) node[label={above:$v_{a}$}]{};
\draw (3,0) node[ground](GND){} to[vsourcesin, l= $10^5 v_{i}$] (3,2);
\draw (3,2) to[R=$10^{2}\ohm$] (5.5,2) node[label={above:$v_{b}$}]{} to[C,l_=$\frac{10^{-9}}{2\pi}F$] (5.5,0) node[ground](GND){};
\draw (5.5,2) to[R=$10^{3}\ohm$] (8,2) node[label={above:$v_{c}$}]{} to[C,l_=$\frac{10^{-9}}{2\pi}F$] (8,0) node[ground](GND){};
\draw (8,2) to[R=$10^{4}\ohm$] (10.5,2) to[C,l_=$\frac{10^{-9}}{2\pi}F$] (10.5,0) node[ground](GND){};
\draw (10.5,2) -- (11.5,2) node[label={above:$v_{o}$}]{};

\end{circuitikz}
}
%	\end{center},
%	\caption{}
%	\label{fig:ee18btech11014_Closed-Loop Equivalent Circuit alpha=90}
%\end{figure}
%
%From the Equivalent Circuit Diagram,
%\begin{align}
%G(s) = \dfrac{10^5}{\left(1+\frac{s}{2\pi 10^{5}}\right)\left(1+\frac{s}{2\pi 10^{6}}\right)\left(1+\frac{s}{2\pi 10^{7}}\right)}\\
%H(s) = \frac{v_{f}}{v_{o}} = 3.162\times 10^{-5}
%\end{align}
%
%The Closed-Loop Gain,
%\begin{align}
%v_{i} = v_{s} - v_{f}\\
%\frac{v_{o}}{G} = v_{s} - Hv_{o}\\
%\frac{v_{o}}{v_{s}} = \frac{G}{1+GH}
%\end{align}
%
%So, the Closed-Loop Gain,
%\begin{align}
%T(s) = \frac{v_{o}}{v_{s}} = \dfrac{10^5}{3.162 + \left(1+s\frac{s}{2\pi 10^{5}}\right)\left(1+\frac{s}{2\pi 10^{6}}\right)\left(1+j\frac{s}{2\pi 10^{7}}\right)}
%\end{align}
\end{enumerate}

\end{document}
