\begin{enumerate}[label=\thesubsection.\arabic*.,ref=\thesubsection.\theenumi]
\numberwithin{equation}{enumi}
\numberwithin{figure}{enumi}

\item Draw the equivalent control system for the feedback current amplifier shown in \ref{fig:ee18btech11014_Input}
\renewcommand{\thefigure}{\theenumi.\arabic{figure}}
\begin{figure}[h!]
	\begin{center}
		\resizebox{\columnwidth/1}{!}{\begin{circuitikz}[american]
\ctikzset{tripoles/mos style/arrows}
\draw  (0,0) node[ground](GND){} -- (0,1) to[isource, l= $I_{s}$] (0,3) -- (2,3) to[R=$R_{F}$, i=$I_{f}$] (4,3) -- (6,3)node[label={right:C}]{} to[R=$R_{m}$] (6,0) node[ground](GND){} (6,0);

\draw (1,5) node[nmos,](Q1){};
\draw (1,3) node[label={below:A}]{} to[short,i=$I_{i}$] (Q1.S);
\draw (Q1.center) node[right]{{$Q_{1}$}};
\draw (Q1.G) -- (0,5) node[ground](GND){};
\draw (1,9) node[vcc](VCC){$V_{CC}$} (1,9);
\draw  (1,6)node[label={left:B}]{} -- (1,6.5) to[R=$R_{D}$, i=$I_{i}$] (1,9);
\draw (Q1.D) -- (1,6);

\draw (6,6)node[pmos,](Q2){};
\draw (Q2.D) to[R=$R_{L}$,i=$I_{o}$] (6,3);
\draw (Q2.center) node[right]{{$Q_{2}$}};
\draw (Q2.G) -- (1,6);
\draw (Q2.S) -- (6,9) node[vcc](VCC){$V_{CC}$} (6,10);
\end{circuitikz}
}
	\end{center}
	\caption{}
	\label{fig:ee18btech11014_Input}
\end{figure}
\\
\solution See Fig. 	\ref{fig:ee18btech11014_Control_System}.

\begin{figure}[ht!]
	\begin{center}
		\resizebox{\columnwidth}{!}{\begin{circuitikz}[american]

\draw (2,2)  node[op amp] (OA) {};
\draw (OA.up) -- ++(0, 0.3) node[vcc]{$+10V$};
\draw (OA.down) -- ++(0,-0.3) node[vee]{$-10V$};
\draw (OA.+) -- (0,1.5) to[vsourcesin, l= $v_{s}$] (0,0) node[ground](GND){};
\draw (OA.-) -- (0,2.5) node[ground, rotate=270](GND){};
\draw (OA.out) -- (3,2) node[label={above:$v_{a}$}]{};
\draw (3,2) to[R=$R_{1}$] (5.5,2) node[label={above:$v_{b}$}]{} to[C,l_=$C_{1}$] (5.5,0) node[ground](GND){};
\draw (5.5,2) to[R=$R_{2}$] (8,2) node[label={above:$v_{c}$}]{} to[C,l_=$C_{2}$] (8,0) node[ground](GND){};
\draw (8,2) to[R=$R_{3}$] (10.5,2) to[C,l_=$C_{3}$] (10.5,0) node[ground](GND){};
\draw (10.5,2) -- (11.5,2) node[label={above:$v_{o}$}]{};

\end{circuitikz}
}
	\end{center}
	\caption{}
	\label{fig:ee18btech11014_Control_System}
\end{figure}
\renewcommand{\thefigure}{\theenumi}
\item For the feedback current amplifier shown in \ref{fig:ee18btech11014_Input}, draw the Small-Signal Model. Neglect the Early effect in $Q_{1}$ and $Q_{2}$.\\
\solution See Fig. 	\ref{fig:ee18btech11014_Small_Signal}.

While drawing a Small-Signal Model, we ground all constant voltage sources and open all constant current sources. All Small-Signal paramters are obtained from DC-Analysis of the circuit. Neglecting Early effect, in Small-Signal Analysis a N-MOSFET is modelled as a Current Source with value of current equal to $g_{m}v_{gs}$ flowing from Drain to Source. Whereas a P-MOSFET is modelled as a Current Source with value of current equal to $g_{m}v_{sg}$ flowing from Source to Drain.
\begin{figure}[h!]
	\begin{center}
		\resizebox{\columnwidth/1}{!}{\begin{circuitikz}[american]
\ctikzset{tripoles/mos style/arrows}
\draw  (0,0) node[ground](GND){} -- (0,1) to[isource, l= $I_{s}$] (0,3) -- (2,3) to[R=$R_{F}$, i=$I_{f}$] (4,3) -- (6,3)node[label={right:C}]{} to[R=$R_{m}$] (6,0) node[ground](GND){} (6,0);

\draw (1,3) node[label={below:A}]{} to[short,i=$I_{i}$] (1,4);
\draw (1,6) node[label={right:B}]{} to[cisource, l= $-g_{m_{1}}v_{A}$] (1,4);
\draw (1,4) to[short, -o] (-1,4) node[label={above:$-$}]{} node[label={below:$S_{1}$}]{};
\draw (-1,5.5) node[label={below:$-v_{A}$}]{};
\draw (-2,6) node[label={below:$G_{1}$}]{} to[short, -o] (-1,6) node[label={below:$+$}]{};
\draw (1,6) to[R=$R_{D}$, i=$I_{i}$] (1,9);
\draw (1,9) node[ground,rotate=180](GND){} (1,9);

\draw (6,7) to[R=$R_{L}$,i=$I_{o}$] (6,3);
\draw (6,8) to[cisource, l= $-g_{m_{2}}v_{B}$] (6,7);
\draw (6,6.5) to[short, -o] (4,6.5) node[label={above:$-$}]{} node[label={below:$G_{2}$}]{};
\draw (4,8) node[label={below:$-v_{B}$}]{};
\draw (3,8.5) node[label={below:$S_{2}$}]{} to[short, -o] (4,8.5) node[label={below:$+$}]{};
\draw (6,7) -- (6,9);
\draw (6,9) node[ground,rotate=180](GND){} (6,9);

\end{circuitikz}
}
	\end{center}
	\caption{Small Signal Model}
	\label{fig:ee18btech11014_Small_Signal}
\end{figure}

%------------------------------------------------------------------------%

%\item Describe how the given circuit is a Negetive Feedback Current Amplifier.\\
%\solution 
%For the feedback to be negative, $I_{f}$ must have the same polarity as $I_{s}$. To ascertain that this is the case, we assume an increase in $I_{s}$ and follow the change around the loop: An increase in $I_{s}$ causes $I_{i}$ to increase and the drain voltage of $Q_{1}$ will increase. Since this voltage is applied to the gate of the p-channel device $Q_{2}$ , its increase will cause $I_{o}$ , the drain current of $Q_{2}$, to decrease. Thus, the voltage across $R_{M}$ will decrease, which will cause $I_{f}$ to increase. This is the same polarity assumed for the initial change in
%$I_{s}$, verifying that the feedback is indeed negative.
%------------------------------------------------------------------------%
%
\item Write all the node/loop equations using KCL/KVL.
\\
\solution From Figs. 	\ref{fig:ee18btech11014_Input} and 	\ref{fig:ee18btech11014_Small_Signal},
%
\begin{align}
\label{eq:ee18btech11014_G_der1}
I_i &= \frac{v_B}{R_D}
\\
\label{eq:ee18btech11014_G_der2}
I_o &= -g_{m_2}v_{B}
\\
\label{eq:ee18btech11014_H_der1}
v_{C} - v_{A} &= -I_fR_F
\\
\label{eq:ee18btech11014_H_der2}
v_C &= \brak{I_o + I_f}R_M
\\
I_i &= g_{m_1}v_{A}
\label{eq:ee18btech11014_vA}
\end{align}
%

\item Find the Expression for the Open-Loop Gain $G$.
\label{prob:ee18btech11014_G}
\\
\solution From \eqref{eq:ee18btech11014_G_der1} and \eqref{eq:ee18btech11014_G_der2},
%
\begin{align}
\label{eq:ee18btech11014_G}
G=\frac{I_{o}}{I_{i}} = -g_{m_2}R_D
\end{align}
%------------------------------------------------------------------------%

\item Find the Expression of the Feedback Factor $H$.
\\
\solution 
\begin{align}
H = \frac{I_{f}}{I_{o}},
\label{eq:ee18btech11014_Hdef}
\end{align}


From \eqref{eq:ee18btech11014_H_der1}
 and \eqref{eq:ee18btech11014_H_der2},
\begin{align}
\brak{I_o + I_f}R_M - v_{A} &= -I_fR_F
\\
\implies \brak{I_o + I_f}R_M + \frac{I_i}{g_{m_1}} &=-I_fR_F
\end{align}
from  \eqref{eq:ee18btech11014_vA}. Dividing by  $I_o $,%
\begin{align}
\implies \brak{1 + H}R_M + \frac{1}{g_{m_1}G} &=-HR_F 
\end{align}
%
upon substituting from \label{eq:ee18btech11014_G}
and \label{eq:ee18btech11014_Hdef}.  Simplifying further, we obtain
%
\begin{align}
\implies H &= \frac{\frac{1}{g_{m_1}g_{m_2}R_D} - R_M}{R_F+R_M}
\\
& \approx  -\frac{ R_M}{R_F+R_M}
\label{eq:ee18btech11014_H}
\end{align}
%
for $R_M \gg \frac{1}{g_{m_1}g_{m_2}R_D}$. 
%
%------------------------------------------------------------------------%
\item Find the Expression for the Closed-Loop Gain $T=\frac{I_{o}}{I_{s}}$. 
\\
\solution 
From \eqref{eq:ee18btech11014_G}
 and \eqref{eq:ee18btech11014_H}, 

\begin{align}
\label{eq:ee18btech11014_T}
T &= \frac{I_{o}}{I_{s}} = \frac{G}{1+GH}\\
&=-\frac{g_{m_{2}} R_{D}}{1+g_{m_{2}} R_{D} /\left(1+\frac{R_{F}}{R_{M}}\right)}
%\\
%\implies T &= -\frac{g_{m_{2}} R_{D}}{1+g_{m_{2}} R_{D} /\left(1+\frac{R_{F}}{R_{M}}\right)}
\end{align}

\end{enumerate}
