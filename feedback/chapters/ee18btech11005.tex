\begin{enumerate}[label=\thesubsection.\arabic*.,ref=\thesubsection.\theenumi]
\numberwithin{equation}{enumi}

\item
Consider a unity feedback system as shown in Fig.  \ref{fig:ee18btech11005}, with an integral compensator $\frac{k}{s}$ and open-loop transfer function
\begin{align}
G(s) = \frac{1}{s^2+3s+2}
\end{align}
where k greater than 0. 
%
Find its closed loop transfer function.
\begin{figure}[!ht]
	\begin{center}
		
		\resizebox{\columnwidth}{!}{\begin{enumerate}[label=\thesubsection.\arabic*.,ref=\thesubsection.\theenumi]
\numberwithin{equation}{enumi}

\item Fig. \ref{fig:ee18btech11005_original_circuit} shows a  non-inverting op-amp configuration   with parameters described in Table \ref{table:ee18btech11005_Input_Table}.  Draw the equivalent control system.
\renewcommand{\thefigure}{\theenumi.\arabic{figure}}
%
\begin{figure}[!ht]
	\begin{center}
		
		\resizebox{\columnwidth}{!}{\begin{circuitikz}
\ctikzset{bipoles/length=1cm}

\draw 
(0, 0) node[op amp] (opamp) {}
(opamp.-) to[R,l_=$R_1$,*-*] (-2, 0.35) to (-2.5, 0.35) to (-2.5, 0.35) node[ground]{}
(opamp.-) --(-0.9,1) to[R=$R_2$] (1,1) -- (1,0) --(2,0) node at(2.3,0){$V_0$}
(opamp.out) to (1.5,0)--(1.5,-0.5) to[R=$R_L$] (1.5,-1.5) to (1.5,-1.5) node[ground]{}
(opamp.+) -- (-0.6,-0.35) to[R =$R_s$,*-*] (-2.6,-0.35) to[V=$V_s$] (-2.6,-2.4) node[ground]{}
;\end{circuitikz}
}
	\end{center}
\caption{}
\label{fig:ee18btech11005_original_circuit}
\end{figure}
%
\begin{table}[!ht]
\centering

%%%%%%%%%%%%%%%%%%%%%%%%%%%%%%%%%%%%%%%%%%%%%%%%%%%%%%%%%%%%%%%%%%%%%%
%%                                                                  %%
%%  This is the header of a LaTeX2e file exported from Gnumeric.    %%
%%                                                                  %%
%%  This file can be compiled as it stands or included in another   %%
%%  LaTeX document. The table is based on the longtable package so  %%
%%  the longtable options (headers, footers...) can be set in the   %%
%%  preamble section below (see PRAMBLE).                           %%
%%                                                                  %%
%%  To include the file in another, the following two lines must be %%
%%  in the including file:                                          %%
%%        \def\inputGnumericTable{}                                 %%
%%  at the beginning of the file and:                               %%
%%        \input{name-of-this-file.tex}                             %%
%%  where the table is to be placed. Note also that the including   %%
%%  file must use the following packages for the table to be        %%
%%  rendered correctly:                                             %%
%%    \usepackage[latin1]{inputenc}                                 %%
%%    \usepackage{color}                                            %%
%%    \usepackage{array}                                            %%
%%    \usepackage{longtable}                                        %%
%%    \usepackage{calc}                                             %%
%%    \usepackage{multirow}                                         %%
%%    \usepackage{hhline}                                           %%
%%    \usepackage{ifthen}                                           %%
%%  optionally (for landscape tables embedded in another document): %%
%%    \usepackage{lscape}                                           %%
%%                                                                  %%
%%%%%%%%%%%%%%%%%%%%%%%%%%%%%%%%%%%%%%%%%%%%%%%%%%%%%%%%%%%%%%%%%%%%%%



%%  This section checks if we are begin input into another file or  %%
%%  the file will be compiled alone. First use a macro taken from   %%
%%  the TeXbook ex 7.7 (suggestion of Han-Wen Nienhuys).            %%
\def\ifundefined#1{\expandafter\ifx\csname#1\endcsname\relax}


%%  Check for the \def token for inputed files. If it is not        %%
%%  defined, the file will be processed as a standalone and the     %%
%%  preamble will be used.                                          %%
\ifundefined{inputGnumericTable}

%%  We must be able to close or not the document at the end.        %%
	\def\gnumericTableEnd{\end{document}}


%%%%%%%%%%%%%%%%%%%%%%%%%%%%%%%%%%%%%%%%%%%%%%%%%%%%%%%%%%%%%%%%%%%%%%
%%                                                                  %%
%%  This is the PREAMBLE. Change these values to get the right      %%
%%  paper size and other niceties.                                  %%
%%                                                                  %%
%%%%%%%%%%%%%%%%%%%%%%%%%%%%%%%%%%%%%%%%%%%%%%%%%%%%%%%%%%%%%%%%%%%%%%

	\documentclass[12pt%
			  %,landscape%
                    ]{report}
       \usepackage[latin1]{inputenc}
       \usepackage{fullpage}
       \usepackage{color}
       \usepackage{array}
       \usepackage{longtable}
       \usepackage{calc}
       \usepackage{multirow}
       \usepackage{hhline}
       \usepackage{ifthen}



%%  End of the preamble for the standalone. The next section is for %%
%%  documents which are included into other LaTeX2e files.          %%
\else

%%  We are not a stand alone document. For a regular table, we will %%
%%  have no preamble and only define the closing to mean nothing.   %%
    \def\gnumericTableEnd{}

%%  If we want landscape mode in an embedded document, comment out  %%
%%  the line above and uncomment the two below. The table will      %%
%%  begin on a new page and run in landscape mode.                  %%
%       \def\gnumericTableEnd{\end{landscape}}
%       \begin{landscape}


%%  End of the else clause for this file being \input.              %%
\fi

%%%%%%%%%%%%%%%%%%%%%%%%%%%%%%%%%%%%%%%%%%%%%%%%%%%%%%%%%%%%%%%%%%%%%%
%%                                                                  %%
%%  The rest is the gnumeric table, except for the closing          %%
%%  statement. Changes below will alter the table's appearance.     %%
%%                                                                  %%
%%%%%%%%%%%%%%%%%%%%%%%%%%%%%%%%%%%%%%%%%%%%%%%%%%%%%%%%%%%%%%%%%%%%%%

\providecommand{\gnumericmathit}[1]{#1} 
%%  Uncomment the next line if you would like your numbers to be in %%
%%  italics if they are italizised in the gnumeric table.           %%
%\renewcommand{\gnumericmathit}[1]{\mathit{#1}}
\providecommand{\gnumericPB}[1]%
{\let\gnumericTemp=\\#1\let\\=\gnumericTemp\hspace{0pt}}
 \ifundefined{gnumericTableWidthDefined}
        \newlength{\gnumericTableWidth}
        \newlength{\gnumericTableWidthComplete}
        \newlength{\gnumericMultiRowLength}
        \global\def\gnumericTableWidthDefined{}
 \fi
%% The following setting protects this code from babel shorthands.  %%
 \ifthenelse{\isundefined{\languageshorthands}}{}{\languageshorthands{english}}
%%  The default table format retains the relative column widths of  %%
%%  gnumeric. They can easily be changed to c, r or l. In that case %%
%%  you may want to comment out the next line and uncomment the one %%
%%  thereafter                                                      %%
\providecommand\gnumbox{\makebox[0pt]}
%%\providecommand\gnumbox[1][]{\makebox}

%% to adjust positions in multirow situations                       %%
\setlength{\bigstrutjot}{\jot}
\setlength{\extrarowheight}{\doublerulesep}

%%  The \setlongtables command keeps column widths the same across  %%
%%  pages. Simply comment out next line for varying column widths.  %%
\setlongtables

\setlength\gnumericTableWidth{%
	53pt+%
	93pt+%
0pt}
\def\gumericNumCols{2}
\setlength\gnumericTableWidthComplete{\gnumericTableWidth+%
         \tabcolsep*\gumericNumCols*2+\arrayrulewidth*\gumericNumCols}
\ifthenelse{\lengthtest{\gnumericTableWidthComplete > \linewidth}}%
         {\def\gnumericScale{\ratio{\linewidth-%
                        \tabcolsep*\gumericNumCols*2-%
                        \arrayrulewidth*\gumericNumCols}%
{\gnumericTableWidth}}}%
{\def\gnumericScale{1}}

%%%%%%%%%%%%%%%%%%%%%%%%%%%%%%%%%%%%%%%%%%%%%%%%%%%%%%%%%%%%%%%%%%%%%%
%%                                                                  %%
%% The following are the widths of the various columns. We are      %%
%% defining them here because then they are easier to change.       %%
%% Depending on the cell formats we may use them more than once.    %%
%%                                                                  %%
%%%%%%%%%%%%%%%%%%%%%%%%%%%%%%%%%%%%%%%%%%%%%%%%%%%%%%%%%%%%%%%%%%%%%%

\ifthenelse{\isundefined{\gnumericColA}}{\newlength{\gnumericColA}}{}\settowidth{\gnumericColA}{\begin{tabular}{@{}p{110pt*\gnumericScale}@{}}x\end{tabular}}
\ifthenelse{\isundefined{\gnumericColB}}{\newlength{\gnumericColB}}{}\settowidth{\gnumericColB}{\begin{tabular}{@{}p{20pt*\gnumericScale}@{}}x\end{tabular}}

\begin{tabular}[c]{%
	b{\gnumericColA}%
	b{\gnumericColB}%
	}

%%%%%%%%%%%%%%%%%%%%%%%%%%%%%%%%%%%%%%%%%%%%%%%%%%%%%%%%%%%%%%%%%%%%%%
%%  The longtable options. (Caption, headers... see Goosens, p.124) %%
%	\caption{The Table Caption.}             \\	%
% \hline	% Across the top of the table.
%%  The rest of these options are table rows which are placed on    %%
%%  the first, last or every page. Use \multicolumn if you want.    %%

%%  Header for the first page.                                      %%
%	\multicolumn{2}{c}{The First Header} \\ \hline 
%	\multicolumn{1}{c}{colTag}	%Column 1
%	&\multicolumn{1}{c}{colTag}	\\ \hline %Last column
%	\endfirsthead

%%  The running header definition.                                  %%
%	\hline
%	\multicolumn{2}{l}{\ldots\small\slshape continued} \\ \hline
%	\multicolumn{1}{c}{colTag}	%Column 1
%	&\multicolumn{1}{c}{colTag}	\\ \hline %Last column
%	\endhead

%%  The running footer definition.                                  %%
%	\hline
%	\multicolumn{2}{r}{\small\slshape continued\ldots} \\
%	\endfoot

%%  The ending footer definition.                                   %%
%	\multicolumn{2}{c}{That's all folks} \\ \hline 
%	\endlastfoot
%%%%%%%%%%%%%%%%%%%%%%%%%%%%%%%%%%%%%%%%%%%%%%%%%%%%%%%%%%%%%%%%%%%%%%

\hhline{|-|-}
	 \multicolumn{1}{|p{\gnumericColA}|}%
	{\gnumericPB{\centering}\gnumbox{\textbf{Parameter}}}
	&\multicolumn{1}{p{\gnumericColB}|}%
	{\gnumericPB{\centering}\gnumbox{\textbf{Value}}}
\\
\hhline{|-|-}
	 \multicolumn{1}{|p{\gnumericColA}|}%
	{\gnumericPB{\centering}\gnumbox{\textbf{input resistance}}}
	&\multicolumn{1}{p{\gnumericColB}|}%
	{\gnumericPB{\centering}\gnumbox{\textbf{$\infty$}}}
\\
\hhline{|-|-}
	 \multicolumn{1}{|p{\gnumericColA}|}%
	{\gnumericPB{\centering}\gnumbox{\textbf{output resistance}}}
	&\multicolumn{1}{p{\gnumericColB}|}%
	{\gnumericPB{\centering}\gnumbox{\textbf{0}}}
\\
\hhline{|-|-}
	 \multicolumn{1}{|p{\gnumericColA}|}%
	{\gnumericPB{\centering}\gnumbox{\textbf{Input voltage}}}
	&\multicolumn{1}{p{\gnumericColB}|}%
	{\gnumericPB{\centering}\gnumbox{\textbf{$V_s$}}}
\\
\hhline{|-|-}
	 \multicolumn{1}{|p{\gnumericColA}|}%
	{\gnumericPB{\centering}\gnumbox{\textbf{Output Voltage}}}
	&\multicolumn{1}{p{\gnumericColB}|}%
	{\gnumericPB{\centering}\gnumbox{\textbf{$V_o$}}}
\\
\hhline{|-|-}
	 \multicolumn{1}{|p{\gnumericColA}|}%
	{\gnumericPB{\centering}\gnumbox{\textbf{Feeding resistance}}}
	&\multicolumn{1}{p{\gnumericColB}|}%
	{\gnumericPB{\centering}\gnumbox{\textbf{$R_1$}}}
\\

\hhline{|-|-}
	 \multicolumn{1}{|p{\gnumericColA}|}%
	{\gnumericPB{\centering}\gnumbox{\textbf{Feedback resistance}}}
	&\multicolumn{1}{p{\gnumericColB}|}%
	{\gnumericPB{\centering}\gnumbox{\textbf{$R_2$}}}
\\
\hhline{|-|-}
	 \multicolumn{1}{|p{\gnumericColA}|}%
	{\gnumericPB{\centering}\gnumbox{\textbf{Source resistance}}}
	&\multicolumn{1}{p{\gnumericColB}|}%
	{\gnumericPB{\centering}\gnumbox{\textbf{$R_s$}}}
\\
\hhline{|-|-}
	 \multicolumn{1}{|p{\gnumericColA}|}%
	{\gnumericPB{\centering}\gnumbox{\textbf{load resistance}}}
	&\multicolumn{1}{p{\gnumericColB}|}%
	{\gnumericPB{\centering}\gnumbox{\textbf{$R_L$}}}
\\

\hhline{|-|-|}
\end{tabular}

\ifthenelse{\isundefined{\languageshorthands}}{}{\languageshorthands{\languagename}}
\gnumericTableEnd

\caption{}
\label{table:ee18btech11005_Input_Table}
\end{table}
\\
\solution  See 	Fig. \ref{fig:ee18btech11005_equivalent_control_system}
\begin{figure}[!ht]
	\begin{center}
			\resizebox{\columnwidth}{!}{
\tikzstyle{block} = [draw, fill=blue!20, rectangle, 
    minimum height=3em, minimum width=6em]
\tikzstyle{sum} = [draw, fill=blue!20, circle, node distance=1cm]
\tikzstyle{input} = [coordinate]
\tikzstyle{output} = [coordinate]
\tikzstyle{pinstyle} = [pin edge={to-,thin,black}]

\begin{tikzpicture}[auto, node distance=2cm,>=latex']
    \node [input, name=input] {$V_s$};
    \node [sum, right of=input] (sum) {};
    \node [block, right of=sum] (controller) {$G$};
    \node [output, right of=controller] (output) {};
    \node [block, below of=controller] (feedback) {$H$};
    \draw [draw,->] (input) -- node {$V_s$} (sum);
    \draw [->] (sum) -- node {$V_i$} (controller);
    \draw [->] (controller) -- node [name=y] {$V_o$}(output);
    \draw [->] (y) |- (feedback);
    \draw [->] (feedback) -| node[pos=0.99]{$-$}  node [near end] {$V_f$} (sum);
\end{tikzpicture}
}
	\end{center}
\caption{}
\label{fig:ee18btech11005_equivalent_control_system}
\end{figure}
\renewcommand{\thefigure}{\theenumi}

\item Draw the small signal model for Fig. \ref{fig:ee18btech11005_original_circuit}.
\\
\solution
The equivalent circuit of the amplifier is in Fig. \ref{fig:ee18btech11005_equivalent_circuit}
\begin{figure}[!ht]
	\begin{center}
		
		\resizebox{\columnwidth}{!}{\usetikzlibrary{decorations.markings}
\begin{circuitikz}
\ctikzset{bipoles/length=1cm}

\draw 
(0, 0) to[V=$V_s$] (0,-1.5) to (0,-1.5) node[ground]{}
(0,0) -- (0,1)--(0.25,1) to[R=$R_s$] (1.5,1)  node at(1.8,1){$+$}
%(1.5,3) node[pos=10]{$V_i$}
(1.5,-1.25)  node at(1.7,-1.25){$-$} 
(1.5,-1.25) -- (1,-1.25) -- (1,-1.75) to[R=$R_1$] (1,-2.75) --(1,-3) node[ground]{}
(1,-1.5) to[R=$R_2$] (5,-1.5){}
(5,-1.5) -- (5,1) --(3.5,1) to[V=$GV_i$] (3.5,-0.5) node[ground]{}
(5,1) --(6,1) to[R=$R_l$,*-*] (6,-0.5) node[ground]{}
(6,1) --(6.5,1) node at(6.8,1){$V_0$}
node at(1.8,-0.3) {$V_i$}
node at(0.6,-1.75){$+$}
node at(0.6,-3){$-$}
node at(0.6,-2.5){$V_f$}
;\end{circuitikz}
}
	\end{center}
\caption{}
\label{fig:ee18btech11005_equivalent_circuit}
\end{figure}

\item Assuming that the operational amplifier has infinite input resistance and zero output resistance, find  the {\em feedback factor} $H$.
\\
\solution From Fig. \ref{fig:ee18btech11005_equivalent_circuit},

\begin{align}
\label{eq:ee18btech11005_opamp_output}
V_0 &= GV_i
\\
 V_i &= V_s -V_f
\\
V_f &= \frac{R_1}{R_1+R_2}V_o
\end{align}
%
assuming that the current through $R_s$ is very small.  Thus, 
\begin{align}
H &=  \frac{V_f}{V_o} = \frac{R_1}{R_1+R_2}
\label{eq:ee18btech11005_H}
\end{align}
\item  Obtain the closed loop gain $T$ and summarize your results through a Table.
\\
\solution Table \ref{table:ee18btech11005_Output_Table} provides a summary.

\begin{align}
\label{eq:ee18btech11005_T}
T &=    \frac{V_0}{V_i}= \frac{G}{1+GH}
  \\
&= \frac{G\brak{R_1+R_2}}{\brak{R_1+R_2}+GR_1}
\end{align}
\begin{table}[!ht]
\centering


%%  This section checks if we are begin input into another file or  %%
%%  the file will be compiled alone. First use a macro taken from   %%
%%  the TeXbook ex 7.7 (suggestion of Han-Wen Nienhuys).            %%
\def\ifundefined#1{\expandafter\ifx\csname#1\endcsname\relax}


%%  Check for the \def token for inputed files. If it is not        %%
%%  defined, the file will be processed as a standalone and the     %%
%%  preamble will be used.                                          %%
\ifundefined{inputGnumericTable}

%%  We must be able to close or not the document at the end.        %%
	\def\gnumericTableEnd{\end{document}}


%%%%%%%%%%%%%%%%%%%%%%%%%%%%%%%%%%%%%%%%%%%%%%%%%%%%%%%%%%%%%%%%%%%%%%
%%                                                                  %%
%%  This is the PREAMBLE. Change these values to get the right      %%
%%  paper size and other niceties.                                  %%
%%                                                                  %%
%%%%%%%%%%%%%%%%%%%%%%%%%%%%%%%%%%%%%%%%%%%%%%%%%%%%%%%%%%%%%%%%%%%%%%

	\documentclass[12pt%
			  %,landscape%
                    ]{report}
       \usepackage[latin1]{inputenc}
       \usepackage{fullpage}
       \usepackage{color}
       \usepackage{array}
       \usepackage{longtable}
       \usepackage{calc}
       \usepackage{multirow}
       \usepackage{hhline}
       \usepackage{ifthen}
%%  End of the preamble for the standalone. The next section is for %%
%%  documents which are included into other LaTeX2e files.          %%
\else

%%  We are not a stand alone document. For a regular table, we will %%
%%  have no preamble and only define the closing to mean nothing.   %%
    \def\gnumericTableEnd{}

%%  If we want landscape mode in an embedded document, comment out  %%
%%  the line above and uncomment the two below. The table will      %%
%%  begin on a new page and run in landscape mode.                  %%
%       \def\gnumericTableEnd{\end{landscape}}
%       \begin{landscape}


%%  End of the else clause for this file being \input.              %%
\fi

%%%%%%%%%%%%%%%%%%%%%%%%%%%%%%%%%%%%%%%%%%%%%%%%%%%%%%%%%%%%%%%%%%%%%%
%%                                                                  %%
%%  The rest is the gnumeric table, except for the closing          %%
%%  statement. Changes below will alter the table's appearance.     %%
%%                                                                  %%
%%%%%%%%%%%%%%%%%%%%%%%%%%%%%%%%%%%%%%%%%%%%%%%%%%%%%%%%%%%%%%%%%%%%%%

\providecommand{\gnumericmathit}[1]{#1} 
%%  Uncomment the next line if you would like your numbers to be in %%
%%  italics if they are italizised in the gnumeric table.           %%
%\renewcommand{\gnumericmathit}[1]{\mathit{#1}}
\providecommand{\gnumericPB}[1]%
{\let\gnumericTemp=\\#1\let\\=\gnumericTemp\hspace{0pt}}
 \ifundefined{gnumericTableWidthDefined}
        \newlength{\gnumericTableWidth}
        \newlength{\gnumericTableWidthComplete}
        \newlength{\gnumericMultiRowLength}
        \global\def\gnumericTableWidthDefined{}
 \fi
%% The following setting protects this code from babel shorthands.  %%
 \ifthenelse{\isundefined{\languageshorthands}}{}{\languageshorthands{english}}
%%  The default table format retains the relative column widths of  %%
%%  gnumeric. They can easily be changed to c, r or l. In that case %%
%%  you may want to comment out the next line and uncomment the one %%
%%  thereafter                                                      %%
\providecommand\gnumbox{\makebox[0pt]}
%%\providecommand\gnumbox[1][]{\makebox}

%% to adjust positions in multirow situations                       %%
\setlength{\bigstrutjot}{\jot}
\setlength{\extrarowheight}{\doublerulesep}

%%  The \setlongtables command keeps column widths the same across  %%
%%  pages. Simply comment out next line for varying column widths.  %%
\setlongtables

\setlength\gnumericTableWidth{%
	50pt+%
	50pt+%
	50pt+%
0pt}
\def\gumericNumCols{3}
\setlength\gnumericTableWidthComplete{\gnumericTableWidth+%
         \tabcolsep*\gumericNumCols*2+\arrayrulewidth*\gumericNumCols}
\ifthenelse{\lengthtest{\gnumericTableWidthComplete > \linewidth}}%
         {\def\gnumericScale{\ratio{\linewidth-%
                        \tabcolsep*\gumericNumCols*2-%
                        \arrayrulewidth*\gumericNumCols}%
{\gnumericTableWidth}}}%
{\def\gnumericScale{1}}

%%%%%%%%%%%%%%%%%%%%%%%%%%%%%%%%%%%%%%%%%%%%%%%%%%%%%%%%%%%%%%%%%%%%%%
%%                                                                  %%
%% The following are the widths of the various columns. We are      %%
%% defining them here because then they are easier to change.       %%
%% Depending on the cell formats we may use them more than once.    %%
%%                                                                  %%
%%%%%%%%%%%%%%%%%%%%%%%%%%%%%%%%%%%%%%%%%%%%%%%%%%%%%%%%%%%%%%%%%%%%%%

\ifthenelse{\isundefined{\gnumericColA}}{\newlength{\gnumericColA}}{}\settowidth{\gnumericColA}{\begin{tabular}{@{}p{50pt*\gnumericScale}@{}}x\end{tabular}}
\ifthenelse{\isundefined{\gnumericColB}}{\newlength{\gnumericColB}}{}\settowidth{\gnumericColB}{\begin{tabular}{@{}p{60pt*\gnumericScale}@{}}x\end{tabular}}
\ifthenelse{\isundefined{\gnumericColC}}{\newlength{\gnumericColC}}{}\settowidth{\gnumericColC}{\begin{tabular}{@{}p{60pt*\gnumericScale}@{}}x\end{tabular}}

\begin{tabular}[c]{%
	b{\gnumericColA}%
	b{\gnumericColB}%
	b{\gnumericColC}%
	}

%%%%%%%%%%%%%%%%%%%%%%%%%%%%%%%%%%%%%%%%%%%%%%%%%%%%%%%%%%%%%%%%%%%%%%
%%  The longtable options. (Caption, headers... see Goosens, p.124) %%
%	\caption{The Table Caption.}             \\	%
% \hline	% Across the top of the table.
%%  The rest of these options are table rows which are placed on    %%
%%  the first, last or every page. Use \multicolumn if you want.    %%

%%  Header for the first page.                                      %%
%	\multicolumn{3}{c}{The First Header} \\ \hline 
%	\multicolumn{1}{c}{colTag}	%Column 1
%	&\multicolumn{1}{c}{colTag}	%Column 2
%	&\multicolumn{1}{c}{colTag}	\\ \hline %Last column
%	\endfirsthead

%%  The running header definition.                                  %%
%	\hline
%	\multicolumn{3}{l}{\ldots\small\slshape continued} \\ \hline
%	\multicolumn{1}{c}{colTag}	%Column 1
%	&\multicolumn{1}{c}{colTag}	%Column 2
%	&\multicolumn{1}{c}{colTag}	\\ \hline %Last column
%	\endhead

%%  The running footer definition.                                  %%
%	\hline
%	\multicolumn{3}{r}{\small\slshape continued\ldots} \\
%	\endfoot

%%  The ending footer definition.                                   %%
%	\multicolumn{3}{c}{That's all folks} \\ \hline 
%	\endlastfoot
%%%%%%%%%%%%%%%%%%%%%%%%%%%%%%%%%%%%%%%%%%%%%%%%%%%%%%%%%%%%%%%%%%%%%%

\hhline{|-|-|-}
	 \multicolumn{1}{|p{\gnumericColA}|}%
	{\gnumericPB{\centering}\textbf{Parameters}}
	&\multicolumn{1}{p{\gnumericColB}|}%
	{\gnumericPB{\centering}\textbf{Definition}}
	&\multicolumn{1}{p{\gnumericColC}|}%
	{\gnumericPB{\centering}\textbf{For given circuit}}

	
\\


\hhline{|---|}
	 \multicolumn{1}{|p{\gnumericColA}|}%
	{\gnumericPB{\centering}{Open loop gain}}
	&\multicolumn{1}{p{\gnumericColB}|}%
	{\gnumericPB{\centering}G}
	&\multicolumn{1}{p{\gnumericColC}|}%
	{\gnumericPB{\centering}G}

\\
\hhline{|---|}
	 \multicolumn{1}{|p{\gnumericColA}|}%
	{\gnumericPB{\centering}Feedback factor}
	&\multicolumn{1}{p{\gnumericColB}|}%
	{\gnumericPB{\centering}H}
	&\multicolumn{1}{p{\gnumericColC}|}%
	{\gnumericPB{\centering}{$\frac{R_1}{R_1+R_2}$}}

\\
\hhline{|---|}
	 \multicolumn{1}{|p{\gnumericColA}|}%
	{\gnumericPB{\centering}Loop gain}
	&\multicolumn{1}{p{\gnumericColB}|}%
	{\gnumericPB{\centering}GH}
	&\multicolumn{1}{p{\gnumericColC}|}%
	{\gnumericPB{\centering}{$G\frac{R_1}{R_1+R_2}$}}

\\


\hhline{|-|-|-|}
    \multicolumn{1}{|p{\gnumericColA}|}%
	{\gnumericPB{\centering}Amount of feedback}
	&\multicolumn{1}{p{\gnumericColB}|}%
	{\gnumericPB{\centering}1+GH}
	&\multicolumn{1}{p{\gnumericColC}|}%
	{\gnumericPB{\centering}{$1+\frac{GR_1}{R_1+R_2}$}}

\\
\hhline{|---|}
	 \multicolumn{1}{|p{\gnumericColA}|}%
	{\gnumericPB{\centering}Closed loop gain }
	&\multicolumn{1}{p{\gnumericColB}|}%
	{\gnumericPB{\centering}{$\frac{G}{1+GH}$}}
	&\multicolumn{1}{p{\gnumericColC}|}%
	{\gnumericPB{\centering}{$\frac{G(R_1+R_2)}{R_1+R_2+GR_1}$}}

\\
\hhline{|-|-|-|}
\end{tabular}

\ifthenelse{\isundefined{\languageshorthands}}{}{\languageshorthands{\languagename}}
\gnumericTableEnd

\caption{}
\label{table:ee18btech11005_Output_Table}
\end{table}
%
\item Find the condition under which closed loop gain T is almost entirely determined by the feedback network.
\\
\solution If 

\begin{align}
 GH &\gg 1,
 \\
T &\approx \frac{1}{H}  = 1 + \frac{R_2}{R_1} 
\label{eq:ee18btech11005_T}
\end{align}
\item If 
\begin{align} 
G & = 10^4
\\
T &= 10,
\end{align}
find $H$.
%$\frac{R_2}{R_1}$.
\\
\solution From Table \ref{table:ee18btech11005_Output_Table}
\begin{align}
    T &=  \frac{G}{1+GH} = 10
\\
\implies  H &= 0.0999
%\frac{R_2}{R_1} &= 9.010
\end{align}
%\item What is the amount of feedback in decibels?
%\solution The value of F in decibals is given by 
%\begin{align}
%    F(dB) &= 20\log\brak{1+GH}\\
%F(dB) &= 60 dB
%\end{align}
\item {\em Gain Desensitivity:} If G decreases by 20\%,what is the corresponding decrease in T?  Comment.
\\
\solution From From Table \ref{table:ee18btech11005_Output_Table},
Given
\begin{align}
T &= \frac{G}{1+GH}
\\
\implies dT &= \frac{dG}{\brak{1+GH}^2}
\\
\implies \frac{dT}{T} &= \frac{1}{1+GH}\frac{dG}{G}
\end{align}
From the information available so far, 
\begin{align}
dG = 20\%, G = 10^4, H = 0.0999
\implies \frac{dT}{T} = 0.025\%
\end{align}
%
using the following code.
\begin{lstlisting}
codes/ee18btech11005/ee18btech11005.py
\end{lstlisting}
%
Thus the closed loop gain is almost invariant to a relatively large (20\%) variation in the open loop gain $G$.  This is known as gain desensitivity.
\end{enumerate}
}
	\end{center}
\caption{}
\label{fig:ee18btech11005}
\end{figure}

\solution $\because H(s) = 1$ in Fig.  \ref{fig:ee18btech11005}, due to unity feedback,   the transfer function is given by
\begin{align}
\frac{Y(s)}{X(s)} &= \frac{G(s)}{1+G(s)H(s)}
\\
\implies T(s) &= \frac{k}{s^3+3s^2+2s}
\end{align}
%
\item Find the {\em characteristic} equation for $G(s)$.
\\
\solution The characteristic equation is
\begin{align}
\label{eq:routh_char_eq}
 1 + G(s)H(s) &= 0 
\\
\implies 1 + \sbrak{\frac{k}{s^3+3s^2+2s}} &= 0
\\
\text{or, } s^3+3s^2+2s+k &= 0
\end{align}
\item Using the tabular method for the Routh hurwitz criterion, find $k > 0$ for which there are two poles of unity feedback system on j${\omega}$ axis.
%
\\
\solution 
This criterion is based on arranging the coefficients of characteristic equation into an array called Routh array.
For any characteristic equation 
\begin{multline}
q(s) = a_0s^n+a_1s^{n-1}+.....+a_{n-1}s+a_n = 0
\end{multline}
the Routh array can be constructed as 
 
\begin{align}
\mydet{s^n\\s^{n-1}\\s^{n-2} \\ \vdots}
 \mydet{a_0 & a_2 & a_4 & \cdots \\
a_1 & a_3 & a_5 & \cdots  \\
b_1 & b_2 & b_3 & \cdots \\
\vdots & \vdots & \vdots & \ddots &\vdots 
 \cdots \\}
\end{align}
%
 where
 \begin{align}
 b_1 =\frac{ a_1a_2-a_0a_3}{a_1}  
 \\
 b_2 =\frac{ a_1a_4-a_0a_5}{a_1} 
 \\
 c_1=\frac{ b_1a_3-a_1b_2}{b_1} 
\\
 c_2=\frac{ b_1a_5-a_1b_3}{b_1}  
\end{align}
For poles to lie on imaginary axis any one entire row of hurwitz matrix should be zero.
Constructing the routh array for the characteristic equation obtained in \ref{eq:routh_char_eq},
%
\begin{align}
 s^3+3s^2+2s+k = 0
\end{align}
%
\begin{align}
\mydet{s^3\\s^2\\s^1 \\ s^0}
\mydet{1 & 2 \\ 3 & k \\  \frac{6-k}{3} & 0\\ k & 0}
\end{align}
For poles on $\j \omega$ axis any one of the row should be zero.
%
\begin{align}
\therefore \frac{6-k}{3} &= 0 \text{ or } k = 0
\\
\implies k &= 6 \quad \because k > 0
\end{align}
\item Repeat the above using the determinant method.
\\
\solution The {\em Routh matrix} can be expressed as
\begin{align}
\vec{R} = \myvec{a_0 & a_2 & a_4 & \cdots \\
a_1 & a_3 & a_5 & \cdots  \\
 0 & a_0 & a_2\cdots \\
 0 & a_1 & a_3 \cdots\\
\vdots & \vdots & \vdots & \ddots &\vdots 
\cdots \\}
\end{align}
and the corresponding Routh determinants are
\begin{align}
D_1 &= |a_0|
\\
D_2 &= 
\mydet{
a_0 & a_2 
\\ 
a_1 & a_3
} 
\\
D_3 &=\mydet{
a_0 & a_2 & a_4 
\\ a_1 & a_3 & a_5 
\\ 0 & a_0 & a_2}
\\
\dots
\end{align}
If at least any one of the Determinents are zero then the poles lie on imaginary axes.  From \eqref{eq:routh_char_eq},
%
\begin{align}
D_1 &= 1 \ne 0
\\
D2 &= \mydet{
1 & 2 \\ 3 & k } 
&= k-6 =0 \implies k = 6
\end{align}
%
\item Verify your answer using a python code for both the determinant method as well as the tabular method.
\label{prob:ee18btech11005_python}
\\
\solution 
The following code verifies the stability using the tabular method 
%
\begin{lstlisting}
codes/ee18btech11005_2.py
\end{lstlisting}
and the following one verifies using the determinant method.
\begin{lstlisting}
codes/ee18btech11005.py
\end{lstlisting}
%
provides the necessary soution.
\begin{itemize}
\item  For the system to be stable all coefficients should lie on left half of s-plane. Because if any pole is in right half of s-plane then there will be a component in output that increases without bound,causing system to be unstable.
All the coefficients in the characteristic equation should be positive.This is necessary condition but not sufficient.Because it may have poles on right half of s plane.
Poles are the roots of the characteristic equation.
    \item A system is stable if all of its characteristic modes go to finite value as t goes to infinity.It is possible only if all the poles are on the left half of s plane.
    The characteristic equation should have negative roots only. So the first column should always be greater than zero.That means no sign changes.
    \item A system is unstable if its characteristic modes are not bounded. Then the characteristic equation will also have roots in the right side of s-plane.That means it has sign changes.
    \end{itemize}

\end{enumerate}


