\tikzstyle{block} = [draw, fill=blue!20, rectangle, 
    minimum height=1cm, minimum width=1cm]
\tikzstyle{sum} = [draw, fill=blue!20, circle, node distance=1cm]
\tikzstyle{input} = [coordinate]
\tikzstyle{output} = [coordinate]
\tikzstyle{pinstyle} = [pin edge={to-,thin,black}]

% The block diagram code is probably more verbose than necessary
\begin{tikzpicture}[auto, node distance=2cm,>=latex']
    % We start by placing the blocks
    \node [input, name=input] {X(s)};
    \node [sum, right of=input] (sum) {};
    \node [block, right of=sum] (controller) {k};
    \node [block, right of=controller] (system) {G(s)};
    % We draw an edge between the controller and system block to 
    % calculate the coordinate u. We need it to place the measurement block. 
    \draw [->] (controller) -- node[name=u] {} (system);
    \node [output, right of=system] (output) {};
    \node [block, below of=u] (measurements) {1};

    % Once the nodes are placed, connecting them is easy. 
    \draw [draw,->] (input) -- node {$X(s)$} (sum);
    \draw [->] (sum) -- node {} (controller);
    \draw [->] (system) -- node [name=y] {$Y(s)$}(output);
    \draw [->] (y) |- (measurements);
    \draw [->] (measurements) -| node[pos=0.99] {$-$} 
        node [near end] {} (sum);
\end{tikzpicture}
