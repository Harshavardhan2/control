%\begin{figure}
\tikzstyle{block} = [draw, fill=blue!20, rectangle, 
    minimum height=1cm, minimum width=1cm]
\tikzstyle{sum} = [draw, fill=blue!20, circle, node distance=1cm]
\tikzstyle{input} = [coordinate]
\tikzstyle{output} = [coordinate]
\tikzstyle{pinstyle} = [pin edge={to-,thin,black}]

% The block diagram code is probably more verbose than necessary
\begin{tikzpicture}[auto, node distance=2cm,>=latex']
    % We start by placing the blocks
    \node [input, name=input] {};
    \node [sum, right of=input] (sum) {};
    \node [block, right of=sum] (controller) {$I = K_{I} \int e(t)$};
    \node [sum, right of=controller, node distance = 2cm] (sum2) {};
    \node [block, right of=sum2,
            node distance=2cm] (system) {Process};
    % We draw an edge between the controller and system block to 
    % calculate the coordinate u. We need it to place the measurement block. 

    \node [output, right of=system] (output) {};
    \node [block, below of=sum2] (measurements) {1};
    \node [block, above of=controller] (second){$P = K_{P} e(t)$};

    % Once the nodes are placed, connecting them is easy. 
    \draw [draw,->] (input) -- node {$r(t)$} (sum);
    \draw [->] (sum) -- node {$e(t)$} (controller);
    \draw [->] (sum) |- node[pos=0.99] {} (second);
    \draw [->] (system) -- node [name=y] {$c(t)$}(output);
    \draw [->] (controller) -- node[pos=0.99] {$+$} node[near end] {} (sum2);
    \draw [->] (second) -| node[pos=0.99] {$+$} node [near end] {} (sum2);
    \draw [->] (sum2) -- node {$u(t)$} (system);
    \draw [->] (y) |- (measurements);
    \draw [->] (measurements) -| node[pos=0.99] {$-$} node [near end] {} (sum);
\end{tikzpicture}
%\end{figure}
