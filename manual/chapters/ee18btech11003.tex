\begin{enumerate}[label=\thesubsection.\arabic*.,ref=\thesubsection.\theenumi]
\numberwithin{equation}{enumi}

\item The Block diagram of a system is illustrated in the figure shown, where $X(s)$ is the input and $Y(s)$ is the output.  Draw the equivalent signal flow graph. 
\renewcommand{\thefigure}{\theenumi.\arabic{figure}}
%
\begin{figure}[!ht]
    \begin{center}
		
		\resizebox{\columnwidth}{!}{\tikzstyle{block} = [draw, rectangle, 
    minimum height=1.25em, minimum width=2.5em]
\tikzstyle{sum} = [draw, circle, node distance=1cm]
\tikzstyle{input} = [coordinate]
\tikzstyle{output} = [coordinate]
\tikzstyle{pinstyle} = [pin edge={to-,thin,black}]


\begin{tikzpicture}[auto, node distance=2.5cm,>=latex']
   
    \node [input, name=input] {};
    \node [sum, right of=input] (sum) {};
    \node [block, right of=sum] (controller) {$G$};
    
    \node [output, right of=controller] (output) {};
    \node [block, below of=controller] (measurements) {$H$};

   \draw [draw,->] (input) -- node[pos=0.99] {$+$} node {$V_{s}$} (sum);
    \draw [->] (sum) -- node {$V_{i}$} (controller);
    \draw [->] (controller) -- node [name=y] {$I_{o}$}(output);
    \draw [->] (y) |- (measurements);
    \draw [->] (measurements) -| node[pos=0.99] {$-$} node [near end] {$V_{f}$} (sum);
\end{tikzpicture}
}
	\end{center}
\caption{Block Diagram}
\label{fig:ee18btech11003_block_diagram}
\end{figure}
\\
\solution The signal flow graph of the block diagram in Fig. \ref{fig:ee18btech11003_block_diagram} is available in Fig. \ref{fig:ee18btech11003_signal_flow}
%
\begin{figure}[!ht]
\begin{center}
		
		\resizebox{\columnwidth}{!}{%signal_flow
\begin{tikzpicture}
[
label revd/.is if=labrev,
amark/.style={
            decoration={             
                        markings,   
                        mark=at position {0.5} with { 
                                    \arrow{stealth},
                                    \iflabrev \node[above] {#1};\else \node[below] {#1};\fi
                        }
            },
            postaction={decorate}
},
terminal/.style 2 args={draw,circle,inner sep=2pt,label={#1:#2}},
]

%Place the nodes
\node[terminal={below}{$X(s)$}] (a) at (0,0) {};
\node[terminal={below left}{$N_1$}] (b) at (2cm,0) {};
\node[terminal={below left}{$N_2$}] (c) at (4cm,0) {};
\node[terminal={[xshift=-4mm]below right}{$N_3$}] (d) at (6cm,0) {};
\node[terminal={below right}{$N_4$}] (e) at (8cm,0) {};
\node[terminal={below left}{$N_5$}] (f) at (11cm,0) {};
\node[terminal={below left}{$Y(s)$}] (g) at (13cm,0) {};
%Draw the connections
\draw[amark=$ $] (a) to (b);
\draw[amark=$ $] (b) to (c);
\draw[amark=$s$] (c) to[bend left=45] (d);
\draw[amark=$1/s$] (e) to (f);
\draw[amark=$ $] (f) to (g);
\draw[amark=$ $] (d) to (e);
\draw[amark=$1/s$] (c) to[bend left=-45] (d);
\draw[amark=$-1$] (e) to[bend left=-45] (b);
\draw[amark=$-1$,label revd] (f) to[bend left=50] (b);
%\draw[amark=$-K/M$,label revd] (e) to[bend right=50] (c);
\end{tikzpicture}
}
	\end{center}
\caption{Signal Flow Graph}
\label{fig:ee18btech11003_signal_flow}
\end{figure}
%
\renewcommand{\thefigure}{\theenumi}
\item Draw all the forward paths in Fig. \ref{fig:ee18btech11003_signal_flow}
and compute the respective gains.
\renewcommand{\thefigure}{\theenumi.\arabic{figure}}
\\
\solution The forward paths are available in Figs. \ref{fig:ee18btech11003_P1}
 and \ref{fig:ee18btech11003_P2}.  The respective gains are
\begin{align}
P_1&=s \brak{\frac{1}{s}}=1
\\
P_2&=(1/s)(1/s)=1/s^2
\end{align}
%
\begin{figure}[!ht]
\begin{center}
		
		\resizebox{\columnwidth}{!}{%P1
\begin{tikzpicture}
[
label revd/.is if=labrev,
amark/.style={
            decoration={             
                        markings,   
                        mark=at position {0.5} with { 
                                    \arrow{stealth},
                                    \iflabrev \node[above] {#1};\else \node[below] {#1};\fi
                        }
            },
            postaction={decorate}
},
terminal/.style 2 args={draw,circle,inner sep=2pt,label={#1:#2}},
]

%Place the nodes
\node[terminal={below}{$X(s)$}] (a) at (0,0) {};
\node[terminal={below left}{$N_1$}] (b) at (2cm,0) {};
\node[terminal={below left}{$N_2$}] (c) at (4cm,0) {};
\node[terminal={[xshift=-4mm]below right}{$N_3$}] (d) at (6cm,0) {};
\node[terminal={below right}{$N_4$}] (e) at (8cm,0) {};
\node[terminal={below left}{$N_5$}] (f) at (11cm,0) {};
\node[terminal={below left}{$Y(s)$}] (g) at (13cm,0) {};
%Draw the connections
\draw[amark=$ $] (a) to (b);
\draw[amark=$ $] (b) to (c);
\draw[amark=$s$] (c) to[bend left=45] (d);
\draw[amark=$1/s$] (e) to (f);
\draw[amark=$ $] (f) to (g);
\draw[amark=$ $] (d) to (e);
\end{tikzpicture}
}
	\end{center}
\caption{$P_1$}
\label{fig:ee18btech11003_P1}
\end{figure}
%
\begin{figure}[!ht]
\begin{center}
		
		\resizebox{\columnwidth}{!}{%P2
\begin{tikzpicture}
[
label revd/.is if=labrev,
amark/.style={
            decoration={             
                        markings,   
                        mark=at position {0.5} with { 
                                    \arrow{stealth},
                                    \iflabrev \node[above] {#1};\else \node[below] {#1};\fi
                        }
            },
            postaction={decorate}
},
terminal/.style 2 args={draw,circle,inner sep=2pt,label={#1:#2}},
]

%Place the nodes
\node[terminal={below}{$X(s)$}] (a) at (0,0) {};
\node[terminal={below left}{$N_1$}] (b) at (2cm,0) {};
\node[terminal={below left}{$N_2$}] (c) at (4cm,0) {};
\node[terminal={[xshift=-4mm]below right}{$N_3$}] (d) at (6cm,0) {};
\node[terminal={below right}{$N_4$}] (e) at (8cm,0) {};
\node[terminal={below left}{$N_5$}] (f) at (11cm,0) {};
\node[terminal={below left}{$Y(s)$}] (g) at (13cm,0) {};
%Draw the connections
\draw[amark=$ $] (a) to (b);
\draw[amark=$ $] (b) to (c);
\draw[amark=$1/s$] (e) to (f);
\draw[amark=$ $] (f) to (g);
\draw[amark=$ $] (d) to (e);
\draw[amark=$1/s$] (c) to[bend left=-45] (d);
\end{tikzpicture}}
	\end{center}
\caption{$P_2$}
\label{fig:ee18btech11003_P2}
\end{figure}
\renewcommand{\thefigure}{\theenumi}
%
\item Draw all the loops in Fig. \ref{fig:ee18btech11003_signal_flow} and calculate the respective gains.
\renewcommand{\thefigure}{\theenumi.\arabic{figure}}
\\
\solution The loops are available in Figs. \ref{fig:ee18btech11003_L1}-\ref{fig:ee18btech11003_L4}
and the corresponding gains are
%
\begin{align}
L_1&=(-1)(s)=-s
\\
L_2&=s\brak{\frac{1}{s}}\brak{-1}=-1
\\
L_3&=\brak{\frac{1}{s}}(-1)=-\frac{1}{s}
\\
L_4&=\brak{\frac{1}{s}}\brak{\frac{1}{s}}(-1)=-\frac{1}{s^2}
\end{align}

\begin{figure}[!ht]
\begin{center}
		
		\resizebox{\columnwidth}{!}{%L1
\begin{tikzpicture}
[
label revd/.is if=labrev,
amark/.style={
            decoration={             
                        markings,   
                        mark=at position {0.5} with { 
                                    \arrow{stealth},
                                    \iflabrev \node[above] {#1};\else \node[below] {#1};\fi
                        }
            },
            postaction={decorate}
},
terminal/.style 2 args={draw,circle,inner sep=2pt,label={#1:#2}},
]

%Place the nodes
\node[terminal={below left}{$N_1$}] (b) at (2cm,0) {};
\node[terminal={below left}{$N_2$}] (c) at (4cm,0) {};
\node[terminal={[xshift=-4mm]below right}{$N_3$}] (d) at (6cm,0) {};
\node[terminal={below right}{$N_4$}] (e) at (8cm,0) {};
%Draw the connections
\draw[amark=$ $] (b) to (c);
\draw[amark=$s$] (c) to[bend left=45] (d);
\draw[amark=$ $] (d) to (e);
\draw[amark=$-1$] (e) to[bend left=-45] (b);
\end{tikzpicture}}
	\end{center}
\caption{$L_1$}
\label{fig:ee18btech11003_L1}
\end{figure}



\begin{figure}[!ht]
\begin{center}
		
		\resizebox{\columnwidth}{!}{%L2
\begin{tikzpicture}
[
label revd/.is if=labrev,
amark/.style={
            decoration={             
                        markings,   
                        mark=at position {0.5} with { 
                                    \arrow{stealth},
                                    \iflabrev \node[above] {#1};\else \node[below] {#1};\fi
                        }
            },
            postaction={decorate}
},
terminal/.style 2 args={draw,circle,inner sep=2pt,label={#1:#2}},
]

%Place the nodes
\node[terminal={below left}{$N_1$}] (b) at (2cm,0) {};
\node[terminal={below left}{$N_2$}] (c) at (4cm,0) {};
\node[terminal={[xshift=-4mm]below right}{$N_3$}] (d) at (6cm,0) {};
\node[terminal={below right}{$N_4$}] (e) at (8cm,0) {};
\node[terminal={below left}{$N_5$}] (f) at (11cm,0) {};
%Draw the connections
\draw[amark=$ $] (b) to (c);
\draw[amark=$s$] (c) to[bend left=45] (d);
\draw[amark=$1/s$] (e) to (f);
\draw[amark=$ $] (d) to (e);
\draw[amark=$-1$,label revd] (f) to[bend left=50] (b);
\end{tikzpicture}}
	\end{center}
\caption{$L_2$}
\label{fig:ee18btech11003_L2}
\end{figure}



\begin{figure}[!ht]
\begin{center}
		
		\resizebox{\columnwidth}{!}{%L3
\begin{tikzpicture}
[
label revd/.is if=labrev,
amark/.style={
            decoration={             
                        markings,   
                        mark=at position {0.5} with { 
                                    \arrow{stealth},
                                    \iflabrev \node[above] {#1};\else \node[below] {#1};\fi
                        }
            },
            postaction={decorate}
},
terminal/.style 2 args={draw,circle,inner sep=2pt,label={#1:#2}},
]

%Place the nodes
\node[terminal={below left}{$N_1$}] (b) at (2cm,0) {};
\node[terminal={below left}{$N_2$}] (c) at (4cm,0) {};
\node[terminal={[xshift=-4mm]below right}{$N_3$}] (d) at (6cm,0) {};
\node[terminal={below right}{$N_4$}] (e) at (8cm,0) {};
%Draw the connections
\draw[amark=$ $] (b) to (c);
\draw[amark=$ $] (d) to (e);
\draw[amark=$1/s$] (c) to[bend left=-45] (d);
\draw[amark=$-1$] (e) to[bend left=-45] (b);
\end{tikzpicture}}
	\end{center}
\caption{$L_3$}
\label{fig:ee18btech11003_L3}
\end{figure}



\begin{figure}[!ht]
\begin{center}
		
		\resizebox{\columnwidth}{!}{%L4
\begin{tikzpicture}
[
label revd/.is if=labrev,
amark/.style={
            decoration={             
                        markings,   
                        mark=at position {0.5} with { 
                                    \arrow{stealth},
                                    \iflabrev \node[above] {#1};\else \node[below] {#1};\fi
                        }
            },
            postaction={decorate}
},
terminal/.style 2 args={draw,circle,inner sep=2pt,label={#1:#2}},
]

%Place the nodes
\node[terminal={below left}{$N_1$}] (b) at (2cm,0) {};
\node[terminal={below left}{$N_2$}] (c) at (4cm,0) {};
\node[terminal={[xshift=-4mm]below right}{$N_3$}] (d) at (6cm,0) {};
\node[terminal={below right}{$N_4$}] (e) at (8cm,0) {};
\node[terminal={below left}{$N_5$}] (f) at (11cm,0) {};
%Draw the connections
\draw[amark=$ $] (b) to (c);
\draw[amark=$1/s$] (e) to (f);
\draw[amark=$ $] (d) to (e);
\draw[amark=$1/s$] (c) to[bend left=-45] (d);
\draw[amark=$-1$,label revd] (f) to[bend left=50] (b);
\end{tikzpicture}}
	\end{center}
\caption{$L_4$}
\label{fig:ee18btech11003_L4}
\end{figure}

\renewcommand{\thefigure}{\theenumi}

\item State Mason's Gain formula and explain the parameters through a table.
\\
\solution 
According to Mason's Gain Formula,
\begin{align}
T &= \frac{Y(s)}{X(s)} 
\\
 &= \frac{\sum_{i=1}^{N} P_i\Delta_i}{\Delta}
\label{eq:ee18btech11003_mason}
\end{align}
%
where the parameters are described in Table \ref{table:ee18btech11003}
\begin{table}[!ht]
\centering
%%%%%%%%%%%%%%%%%%%%%%%%%%%%%%%%%%%%%%%%%%%%%%%%%%%%%%%%%%%%%%%%%%%%%%
%%                                                                  %%
%%  This is the header of a LaTeX2e file exported from Gnumeric.    %%
%%                                                                  %%
%%  This file can be compiled as it stands or included in another   %%
%%  LaTeX document. The table is based on the longtable package so  %%
%%  the longtable options (headers, footers...) can be set in the   %%
%%  preamble section below (see PRAMBLE).                           %%
%%                                                                  %%
%%  To include the file in another, the following two lines must be %%
%%  in the including file:                                          %%
%%        \def\inputGnumericTable{}                                 %%
%%  at the beginning of the file and:                               %%
%%        \input{name-of-this-file.tex}                             %%
%%  where the table is to be placed. Note also that the including   %%
%%  file must use the following packages for the table to be        %%
%%  rendered correctly:                                             %%
%%    \usepackage[latin1]{inputenc}                                 %%
%%    \usepackage{color}                                            %%
%%    \usepackage{array}                                            %%
%%    \usepackage{longtable}                                        %%
%%    \usepackage{calc}                                             %%
%%    \usepackage{multirow}                                         %%
%%    \usepackage{hhline}                                           %%
%%    \usepackage{ifthen}                                           %%
%%  optionally (for landscape tables embedded in another document): %%
%%    \usepackage{lscape}                                           %%
%%                                                                  %%
%%%%%%%%%%%%%%%%%%%%%%%%%%%%%%%%%%%%%%%%%%%%%%%%%%%%%%%%%%%%%%%%%%%%%%



%%  This section checks if we are begin input into another file or  %%
%%  the file will be compiled alone. First use a macro taken from   %%
%%  the TeXbook ex 7.7 (suggestion of Han-Wen Nienhuys).            %%
\def\ifundefined#1{\expandafter\ifx\csname#1\endcsname\relax}


%%  Check for the \def token for inputed files. If it is not        %%
%%  defined, the file will be processed as a standalone and the     %%
%%  preamble will be used.                                          %%
\ifundefined{inputGnumericTable}

%%  We must be able to close or not the document at the end.        %%
	\def\gnumericTableEnd{\end{document}}


%%%%%%%%%%%%%%%%%%%%%%%%%%%%%%%%%%%%%%%%%%%%%%%%%%%%%%%%%%%%%%%%%%%%%%
%%                                                                  %%
%%  This is the PREAMBLE. Change these values to get the right      %%
%%  paper size and other niceties.                                  %%
%%                                                                  %%
%%%%%%%%%%%%%%%%%%%%%%%%%%%%%%%%%%%%%%%%%%%%%%%%%%%%%%%%%%%%%%%%%%%%%%

	\documentclass[12pt%
			  %,landscape%
                    ]{report}
       \usepackage[latin1]{inputenc}
       \usepackage{fullpage}
       \usepackage{color}
       \usepackage{array}
       \usepackage{longtable}
       \usepackage{calc}
       \usepackage{multirow}
       \usepackage{hhline}
       \usepackage{ifthen}

	\begin{document}


%%  End of the preamble for the standalone. The next section is for %%
%%  documents which are included into other LaTeX2e files.          %%
\else

%%  We are not a stand alone document. For a regular table, we will %%
%%  have no preamble and only define the closing to mean nothing.   %%
    \def\gnumericTableEnd{}

%%  If we want landscape mode in an embedded document, comment out  %%
%%  the line above and uncomment the two below. The table will      %%
%%  begin on a new page and run in landscape mode.                  %%
%       \def\gnumericTableEnd{\end{landscape}}
%       \begin{landscape}


%%  End of the else clause for this file being \input.              %%
\fi

%%%%%%%%%%%%%%%%%%%%%%%%%%%%%%%%%%%%%%%%%%%%%%%%%%%%%%%%%%%%%%%%%%%%%%
%%                                                                  %%
%%  The rest is the gnumeric table, except for the closing          %%
%%  statement. Changes below will alter the table's appearance.     %%
%%                                                                  %%
%%%%%%%%%%%%%%%%%%%%%%%%%%%%%%%%%%%%%%%%%%%%%%%%%%%%%%%%%%%%%%%%%%%%%%

\providecommand{\gnumericmathit}[1]{#1} 
%%  Uncomment the next line if you would like your numbers to be in %%
%%  italics if they are italizised in the gnumeric table.           %%
%\renewcommand{\gnumericmathit}[1]{\mathit{#1}}
\providecommand{\gnumericPB}[1]%
{\let\gnumericTemp=\\#1\let\\=\gnumericTemp\hspace{0pt}}
 \ifundefined{gnumericTableWidthDefined}
        \newlength{\gnumericTableWidth}
        \newlength{\gnumericTableWidthComplete}
        \newlength{\gnumericMultiRowLength}
        \global\def\gnumericTableWidthDefined{}
 \fi
%% The following setting protects this code from babel shorthands.  %%
 \ifthenelse{\isundefined{\languageshorthands}}{}{\languageshorthands{english}}
%%  The default table format retains the relative column widths of  %%
%%  gnumeric. They can easily be changed to c, r or l. In that case %%
%%  you may want to comment out the next line and uncomment the one %%
%%  thereafter                                                      %%
\providecommand\gnumbox{\makebox[0pt]}
%%\providecommand\gnumbox[1][]{\makebox}

%% to adjust positions in multirow situations                       %%
\setlength{\bigstrutjot}{\jot}
\setlength{\extrarowheight}{\doublerulesep}

%%  The \setlongtables command keeps column widths the same across  %%
%%  pages. Simply comment out next line for varying column widths.  %%
\setlongtables

\setlength\gnumericTableWidth{%
	23pt+%
	190pt+%
	1pt+%
0pt}
\def\gumericNumCols{3}
\setlength\gnumericTableWidthComplete{\gnumericTableWidth+%
         \tabcolsep*\gumericNumCols*2+\arrayrulewidth*\gumericNumCols}
\ifthenelse{\lengthtest{\gnumericTableWidthComplete > \linewidth}}%
         {\def\gnumericScale{\ratio{\linewidth-%
                        \tabcolsep*\gumericNumCols*2-%
                        \arrayrulewidth*\gumericNumCols}%
{\gnumericTableWidth}}}%
{\def\gnumericScale{1}}

%%%%%%%%%%%%%%%%%%%%%%%%%%%%%%%%%%%%%%%%%%%%%%%%%%%%%%%%%%%%%%%%%%%%%%
%%                                                                  %%
%% The following are the widths of the various columns. We are      %%
%% defining them here because then they are easier to change.       %%
%% Depending on the cell formats we may use them more than once.    %%
%%                                                                  %%
%%%%%%%%%%%%%%%%%%%%%%%%%%%%%%%%%%%%%%%%%%%%%%%%%%%%%%%%%%%%%%%%%%%%%%

\ifthenelse{\isundefined{\gnumericColA}}{\newlength{\gnumericColA}}{}\settowidth{\gnumericColA}{\begin{tabular}{@{}p{23pt*\gnumericScale}@{}}x\end{tabular}}
\ifthenelse{\isundefined{\gnumericColB}}{\newlength{\gnumericColB}}{}\settowidth{\gnumericColB}{\begin{tabular}{@{}p{190pt*\gnumericScale}@{}}x\end{tabular}}
\ifthenelse{\isundefined{\gnumericColC}}{\newlength{\gnumericColC}}{}\settowidth{\gnumericColC}{\begin{tabular}{@{}p{1pt*\gnumericScale}@{}}x\end{tabular}}

{\small
\begin{tabular}[c]{%
	b{\gnumericColA}%
	b{\gnumericColB}%
	b{\gnumericColC}%
	}

%%%%%%%%%%%%%%%%%%%%%%%%%%%%%%%%%%%%%%%%%%%%%%%%%%%%%%%%%%%%%%%%%%%%%%
%%  The longtable options. (Caption, headers... see Goosens, p.124) %%
%	\caption{The Table Caption.}             \\	%
% \hline	% Across the top of the table.
%%  The rest of these options are table rows which are placed on    %%
%%  the first, last or every page. Use \multicolumn if you want.    %%

%%  Header for the first page.                                      %%
%	\multicolumn{3}{c}{The First Header} \\ \hline 
%	\multicolumn{1}{c}{colTag}	%Column 1
%	&\multicolumn{1}{c}{colTag}	%Column 2
%	&\multicolumn{1}{c}{colTag}	\\ \hline %Last column
%	\endfirsthead

%%  The running header definition.                                  %%
%	\hline
%	\multicolumn{3}{l}{\ldots\small\slshape continued} \\ \hline
%	\multicolumn{1}{c}{colTag}	%Column 1
%	&\multicolumn{1}{c}{colTag}	%Column 2
%	&\multicolumn{1}{c}{colTag}	\\ \hline %Last column
%	\endhead

%%  The running footer definition.                                  %%
%	\hline
%	\multicolumn{3}{r}{\small\slshape continued\ldots} \\
%	\endfoot

%%  The ending footer definition.                                   %%
%	\multicolumn{3}{c}{That's all folks} \\ \hline 
%	\endlastfoot
%%%%%%%%%%%%%%%%%%%%%%%%%%%%%%%%%%%%%%%%%%%%%%%%%%%%%%%%%%%%%%%%%%%%%%

\hhline{|-|-~}
	 \multicolumn{1}{|p{\gnumericColA}|}%
	{\gnumericPB{\centering}\gnumbox{\textbf{Variable}}}
	&\multicolumn{1}{p{\gnumericColB}|}%
	{\gnumericPB{\centering}\gnumbox{\textbf{Description}}}
	&
\\
\hhline{|--|~}
	 \multicolumn{1}{|p{\gnumericColA}|}%
	{\gnumericPB{\centering}\gnumbox{$P_i$}}
	&\multicolumn{1}{p{\gnumericColB}|}%
	{\gnumericPB{\centering}\gnumbox{$i$th forward path}}
	&
\\
\hhline{|--|~}
	 \multicolumn{1}{|p{\gnumericColA}|}%
	{\gnumericPB{\centering}\gnumbox{$L_j$}}
	&\multicolumn{1}{p{\gnumericColB}|}%
	{\gnumericPB{\centering}\gnumbox{$j$th loop}}
	&
\\
\hhline{|--|~}
	 \multicolumn{1}{|p{\gnumericColA}|}%
	{\gnumericPB{\centering}\gnumbox{$\Delta$}}
	&\multicolumn{1}{p{\gnumericColB}|}%
	{\gnumericPB{\centering}\gnumbox{$1-\sum_{L_i}+\sum_{L_i\cap L_j = \phi}L_iL_j-\sum_{L_i\cap L_j\cap L_k = \phi}L_iL_jL_k+\dots$}}
	&
\\
\hhline{|--|~}
	 \multicolumn{1}{|p{\gnumericColA}|}%
	{\gnumericPB{\centering}\gnumbox{$\Delta_i$}}
	&\multicolumn{1}{p{\gnumericColB}|}%
	{\gnumericPB{\centering}\gnumbox{$1-\sum_{L_k \cap P_i = \phi}L_k +\sum_{L_k\cap L_j \cap P_i = \phi}L_kL_j-\dots$}}
	&
\\
\hhline{|-|-|~}
\end{tabular}
}
\ifthenelse{\isundefined{\languageshorthands}}{}{\languageshorthands{\languagename}}
\gnumericTableEnd

\caption{}
\label{table:ee18btech11003}
\end{table}
\item List the parameters in Table \ref{table:ee18btech11003}
for Fig. \ref{fig:ee18btech11003_signal_flow}.
\\
\solution The parameters are available in Table \ref{table:ee18btech11003_ex}

\begin{table}[!ht]
\centering
%%%%%%%%%%%%%%%%%%%%%%%%%%%%%%%%%%%%%%%%%%%%%%%%%%%%%%%%%%%%%%%%%%%%%%
%%                                                                  %%
%%  This is the header of a LaTeX2e file exported from Gnumeric.    %%
%%                                                                  %%
%%  This file can be compiled as it stands or included in another   %%
%%  LaTeX document. The table is based on the longtable package so  %%
%%  the longtable options (headers, footers...) can be set in the   %%
%%  preamble section below (see PRAMBLE).                           %%
%%                                                                  %%
%%  To include the file in another, the following two lines must be %%
%%  in the including file:                                          %%
%%        \def\inputGnumericTable{}                                 %%
%%  at the beginning of the file and:                               %%
%%        \input{name-of-this-file.tex}                             %%
%%  where the table is to be placed. Note also that the including   %%
%%  file must use the following packages for the table to be        %%
%%  rendered correctly:                                             %%
%%    \usepackage[latin1]{inputenc}                                 %%
%%    \usepackage{color}                                            %%
%%    \usepackage{array}                                            %%
%%    \usepackage{longtable}                                        %%
%%    \usepackage{calc}                                             %%
%%    \usepackage{multirow}                                         %%
%%    \usepackage{hhline}                                           %%
%%    \usepackage{ifthen}                                           %%
%%  optionally (for landscape tables embedded in another document): %%
%%    \usepackage{lscape}                                           %%
%%                                                                  %%
%%%%%%%%%%%%%%%%%%%%%%%%%%%%%%%%%%%%%%%%%%%%%%%%%%%%%%%%%%%%%%%%%%%%%%



%%  This section checks if we are begin input into another file or  %%
%%  the file will be compiled alone. First use a macro taken from   %%
%%  the TeXbook ex 7.7 (suggestion of Han-Wen Nienhuys).            %%
\def\ifundefined#1{\expandafter\ifx\csname#1\endcsname\relax}


%%  Check for the \def token for inputed files. If it is not        %%
%%  defined, the file will be processed as a standalone and the     %%
%%  preamble will be used.                                          %%
\ifundefined{inputGnumericTable}

%%  We must be able to close or not the document at the end.        %%
	\def\gnumericTableEnd{\end{document}}


%%%%%%%%%%%%%%%%%%%%%%%%%%%%%%%%%%%%%%%%%%%%%%%%%%%%%%%%%%%%%%%%%%%%%%
%%                                                                  %%
%%  This is the PREAMBLE. Change these values to get the right      %%
%%  paper size and other niceties.                                  %%
%%                                                                  %%
%%%%%%%%%%%%%%%%%%%%%%%%%%%%%%%%%%%%%%%%%%%%%%%%%%%%%%%%%%%%%%%%%%%%%%

	\documentclass[12pt%
			  %,landscape%
                    ]{report}
       \usepackage[latin1]{inputenc}
       \usepackage{fullpage}
       \usepackage{color}
       \usepackage{array}
       \usepackage{longtable}
       \usepackage{calc}
       \usepackage{multirow}
       \usepackage{hhline}
       \usepackage{ifthen}

	\begin{document}


%%  End of the preamble for the standalone. The next section is for %%
%%  documents which are included into other LaTeX2e files.          %%
\else

%%  We are not a stand alone document. For a regular table, we will %%
%%  have no preamble and only define the closing to mean nothing.   %%
    \def\gnumericTableEnd{}

%%  If we want landscape mode in an embedded document, comment out  %%
%%  the line above and uncomment the two below. The table will      %%
%%  begin on a new page and run in landscape mode.                  %%
%       \def\gnumericTableEnd{\end{landscape}}
%       \begin{landscape}


%%  End of the else clause for this file being \input.              %%
\fi

%%%%%%%%%%%%%%%%%%%%%%%%%%%%%%%%%%%%%%%%%%%%%%%%%%%%%%%%%%%%%%%%%%%%%%
%%                                                                  %%
%%  The rest is the gnumeric table, except for the closing          %%
%%  statement. Changes below will alter the table's appearance.     %%
%%                                                                  %%
%%%%%%%%%%%%%%%%%%%%%%%%%%%%%%%%%%%%%%%%%%%%%%%%%%%%%%%%%%%%%%%%%%%%%%

\providecommand{\gnumericmathit}[1]{#1} 
%%  Uncomment the next line if you would like your numbers to be in %%
%%  italics if they are italizised in the gnumeric table.           %%
%\renewcommand{\gnumericmathit}[1]{\mathit{#1}}
\providecommand{\gnumericPB}[1]%
{\let\gnumericTemp=\\#1\let\\=\gnumericTemp\hspace{0pt}}
 \ifundefined{gnumericTableWidthDefined}
        \newlength{\gnumericTableWidth}
        \newlength{\gnumericTableWidthComplete}
        \newlength{\gnumericMultiRowLength}
        \global\def\gnumericTableWidthDefined{}
 \fi
%% The following setting protects this code from babel shorthands.  %%
 \ifthenelse{\isundefined{\languageshorthands}}{}{\languageshorthands{english}}
%%  The default table format retains the relative column widths of  %%
%%  gnumeric. They can easily be changed to c, r or l. In that case %%
%%  you may want to comment out the next line and uncomment the one %%
%%  thereafter                                                      %%
\providecommand\gnumbox{\makebox[0pt]}
%%\providecommand\gnumbox[1][]{\makebox}

%% to adjust positions in multirow situations                       %%
\setlength{\bigstrutjot}{\jot}
\setlength{\extrarowheight}{\doublerulesep}

%%  The \setlongtables command keeps column widths the same across  %%
%%  pages. Simply comment out next line for varying column widths.  %%
\setlongtables

\setlength\gnumericTableWidth{%
	15pt+%
	20pt+%
	39pt+%
	35pt+%
	40pt+%
0pt}
\def\gumericNumCols{5}
\setlength\gnumericTableWidthComplete{\gnumericTableWidth+%
         \tabcolsep*\gumericNumCols*2+\arrayrulewidth*\gumericNumCols}
\ifthenelse{\lengthtest{\gnumericTableWidthComplete > \linewidth}}%
         {\def\gnumericScale{\ratio{\linewidth-%
                        \tabcolsep*\gumericNumCols*2-%
                        \arrayrulewidth*\gumericNumCols}%
{\gnumericTableWidth}}}%
{\def\gnumericScale{1}}

%%%%%%%%%%%%%%%%%%%%%%%%%%%%%%%%%%%%%%%%%%%%%%%%%%%%%%%%%%%%%%%%%%%%%%
%%                                                                  %%
%% The following are the widths of the various columns. We are      %%
%% defining them here because then they are easier to change.       %%
%% Depending on the cell formats we may use them more than once.    %%
%%                                                                  %%
%%%%%%%%%%%%%%%%%%%%%%%%%%%%%%%%%%%%%%%%%%%%%%%%%%%%%%%%%%%%%%%%%%%%%%

\ifthenelse{\isundefined{\gnumericColA}}{\newlength{\gnumericColA}}{}\settowidth{\gnumericColA}{\begin{tabular}{@{}p{15pt*\gnumericScale}@{}}x\end{tabular}}
\ifthenelse{\isundefined{\gnumericColB}}{\newlength{\gnumericColB}}{}\settowidth{\gnumericColB}{\begin{tabular}{@{}p{20pt*\gnumericScale}@{}}x\end{tabular}}
\ifthenelse{\isundefined{\gnumericColC}}{\newlength{\gnumericColC}}{}\settowidth{\gnumericColC}{\begin{tabular}{@{}p{39pt*\gnumericScale}@{}}x\end{tabular}}
\ifthenelse{\isundefined{\gnumericColD}}{\newlength{\gnumericColD}}{}\settowidth{\gnumericColD}{\begin{tabular}{@{}p{35pt*\gnumericScale}@{}}x\end{tabular}}
\ifthenelse{\isundefined{\gnumericColE}}{\newlength{\gnumericColE}}{}\settowidth{\gnumericColE}{\begin{tabular}{@{}p{40pt*\gnumericScale}@{}}x\end{tabular}}

{\small
\begin{tabular}[c]{%
	b{\gnumericColA}%
	b{\gnumericColB}%
	b{\gnumericColC}%
	b{\gnumericColD}%
	b{\gnumericColE}%
	}

%%%%%%%%%%%%%%%%%%%%%%%%%%%%%%%%%%%%%%%%%%%%%%%%%%%%%%%%%%%%%%%%%%%%%%
%%  The longtable options. (Caption, headers... see Goosens, p.124) %%
%	\caption{The Table Caption.}             \\	%
% \hline	% Across the top of the table.
%%  The rest of these options are table rows which are placed on    %%
%%  the first, last or every page. Use \multicolumn if you want.    %%

%%  Header for the first page.                                      %%
%	\multicolumn{5}{c}{The First Header} \\ \hline 
%	\multicolumn{1}{c}{colTag}	%Column 1
%	&\multicolumn{1}{c}{colTag}	%Column 2
%	&\multicolumn{1}{c}{colTag}	%Column 3
%	&\multicolumn{1}{c}{colTag}	%Column 4
%	&\multicolumn{1}{c}{colTag}	\\ \hline %Last column
%	\endfirsthead

%%  The running header definition.                                  %%
%	\hline
%	\multicolumn{5}{l}{\ldots\small\slshape continued} \\ \hline
%	\multicolumn{1}{c}{colTag}	%Column 1
%	&\multicolumn{1}{c}{colTag}	%Column 2
%	&\multicolumn{1}{c}{colTag}	%Column 3
%	&\multicolumn{1}{c}{colTag}	%Column 4
%	&\multicolumn{1}{c}{colTag}	\\ \hline %Last column
%	\endhead

%%  The running footer definition.                                  %%
%	\hline
%	\multicolumn{5}{r}{\small\slshape continued\ldots} \\
%	\endfoot

%%  The ending footer definition.                                   %%
%	\multicolumn{5}{c}{That's all folks} \\ \hline 
%	\endlastfoot
%%%%%%%%%%%%%%%%%%%%%%%%%%%%%%%%%%%%%%%%%%%%%%%%%%%%%%%%%%%%%%%%%%%%%%

\hhline{|-|-|-|-|-}
	 \multicolumn{1}{|p{\gnumericColA}|}%
	{\gnumericPB{\centering}\gnumbox{\textbf{Path}}}
	&\multicolumn{1}{p{\gnumericColB}|}%
	{\gnumericPB{\centering}\gnumbox{\textbf{Value}}}
	&\multicolumn{1}{p{\gnumericColC}|}%
	{\gnumericPB{\centering}\gnumbox{\textbf{Parameter}}}
	&\multicolumn{1}{p{\gnumericColD}|}%
	{\gnumericPB{\raggedright}\gnumbox[l]{\textbf{Value}}}
	&\multicolumn{1}{p{\gnumericColE}|}%
	{\gnumericPB{\raggedright}\gnumbox[l]{\textbf{Remarks}}}
\\
\hhline{|-----|}
	 \multicolumn{1}{|p{\gnumericColA}|}%
	{\gnumericPB{\centering}\gnumbox{$P_1$}}
	&\multicolumn{1}{p{\gnumericColB}|}%
	{1}
	&\multicolumn{1}{p{\gnumericColC}|}%
	{\gnumericPB{\centering}\gnumbox{$\Delta_1$}}
	&\multicolumn{1}{p{\gnumericColD}|}%
	{$1$}
	&\multicolumn{1}{p{\gnumericColE}|}%
	{All loops intersect with $P_1$}
\\
\hhline{|-----|}
	 \multicolumn{1}{|p{\gnumericColA}|}%
	{\gnumericPB{\centering}\gnumbox{$P_2$}}
	&\multicolumn{1}{p{\gnumericColB}|}%
	{$\frac{1}{s^2}$}
	&\multicolumn{1}{p{\gnumericColC}|}%
	{\gnumericPB{\centering}\gnumbox{$\Delta_2$}}
	&\multicolumn{1}{p{\gnumericColD}|}%
	{1}
	&\multicolumn{1}{p{\gnumericColE}|}%
	{All loops intersect with $P_2$}
\\
\hhline{|-----|}
	 \multicolumn{1}{|p{\gnumericColA}|}%
	{\gnumericPB{\centering}\gnumbox{$L_1$}}
	&\multicolumn{1}{p{\gnumericColB}|}%
	{$-s$}
	&\multicolumn{1}{p{\gnumericColC}|}%
	{\setlength{\gnumericMultiRowLength}{0pt}%
	 \addtolength{\gnumericMultiRowLength}{\gnumericColC}%
	 \multirow{4}[2]{\gnumericMultiRowLength}{\parbox{\gnumericMultiRowLength}{%
	 \gnumericPB{\centering}$\Delta$}}}
	&\multicolumn{1}{p{\gnumericColD}|}%
	{\setlength{\gnumericMultiRowLength}{0pt}%
	 \addtolength{\gnumericMultiRowLength}{\gnumericColD}%
	 \multirow{4}[2]{\gnumericMultiRowLength}{%
	 }}
	&\multicolumn{1}{p{\gnumericColE}|}%
	{\setlength{\gnumericMultiRowLength}{0pt}%
	 \addtolength{\gnumericMultiRowLength}{\gnumericColE}%
	 \multirow{4}[2]{\gnumericMultiRowLength}{%
	 }}
\\
\hhline{|--|~~~}
	 \multicolumn{1}{|p{\gnumericColA}|}%
	{\gnumericPB{\centering}\gnumbox{$L_2$}}
	&\multicolumn{1}{p{\gnumericColB}|}%
	{$-1$}
	&\multicolumn{1}{p{\gnumericColC}|}%
	{}
	&\multicolumn{1}{p{\gnumericColD}|}%
	{}
	&\multicolumn{1}{p{\gnumericColE}|}%
	{}
\\
\hhline{|--|~~~}
	 \multicolumn{1}{|p{\gnumericColA}|}%
	{\gnumericPB{\centering}\gnumbox{$L_3$}}
	&\multicolumn{1}{p{\gnumericColB}|}%
	{-$\frac{1}{s}$}
	&\multicolumn{1}{p{\gnumericColC}|}%
	{}
	&\multicolumn{1}{p{\gnumericColD}|}%
	{$1-\sum_{i}L_i$}
	&\multicolumn{1}{p{\gnumericColE}|}%
	{All loops intersect}
\\
\hhline{|--|~~~}
	 \multicolumn{1}{|p{\gnumericColA}|}%
	{\gnumericPB{\centering}\gnumbox{$L_4$}}
	&\multicolumn{1}{p{\gnumericColB}|}%
	{-$\frac{1}{s^2}$}
	&\multicolumn{1}{p{\gnumericColC}|}%
	{}
	&\multicolumn{1}{p{\gnumericColD}|}%
	{}
	&\multicolumn{1}{p{\gnumericColE}|}%
	{}
\\
\hhline{|-|-|-|-|-|}
\end{tabular}
}
\ifthenelse{\isundefined{\languageshorthands}}{}{\languageshorthands{\languagename}}
\gnumericTableEnd

\caption{}
\label{table:ee18btech11003_ex}
\end{table}

\item  Find the transfer function using Mason's Gain Formula.
\renewcommand{\thefigure}{\theenumi.\arabic{figure}}
%
\\
\solution From \eqref{eq:ee18btech11003_mason} and \ref{table:ee18btech11003_ex},
\begin{align}
T(s)&=\frac{P_1 \Delta_1+P_2 \Delta_2}{\Delta}
\\
&=\frac{1 +\frac{1}{s^2}}{1-(-s-1-\frac{1}{s}-\frac{1}{s^2})}
\\
&=\frac{s^2+1}{s^3+2s^2+s+1}
\label{eq:ee18btech11003_gain_sol}
\end{align}
%
after simplification.
\renewcommand{\thefigure}{\theenumi}
%\item Write a program to compute Mason's gain formula, given the branch nodes and gains for each path.
\end{enumerate}
