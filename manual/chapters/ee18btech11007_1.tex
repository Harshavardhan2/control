\begin{enumerate}[label=\thesubsection.\arabic*.,ref=\thesubsection.\theenumi]
\numberwithin{equation}{enumi}

\item Write the transition equations in Fig. \ref{fig:ee18btech11003_signal_flow}.
\\
\solution The equations are
\begin{align}
N_1&=X(s)-N_4-N_5
\\
N_2&=N_1
\\
N_3&=N_2(s+1/s)
\\
N_4&=N_3
\\
Y(s) = N_5&=N_4/s
\label{eq:ee18btech11007_state}
\end{align}
%
\item Obtain the state transition matrix from \eqref{eq:ee18btech11007_state}
\\
\solution The state transition matrix is
\begin{align}
\vec{T} = \myvec{0 & 0 & 0 & -1 & -1 \ \\
1 & 0 & 0 & 0 & 0   \\
 0 & s+1/s & 0 & 0 & 0 \\
 0 & 0 & 1 & 0 & 0 \\
0 & 0 & 0 & 1/s & 0 }
\label{eq:ee18btech11007_trans_mat}
\end{align}
%
\item State the equivalent matrix form of  Mason's gain formula 
\label{prob:ee18btech11007_mat_form}
\\
\solution Let
\begin{align}
    \vec{U}=\brak{\vec{I}-\vec{T}}^{-1}
\label{eq:ee18btech11007_u_mat}
\end{align}
%
The gain from node m to node n of the graph is $U_{mn}$
\item Find the transfer function for the sytem in Fig. \ref{fig:ee18btech11003_block_diagram}
\\
\solution From \eqref{eq:ee18btech11007_trans_mat} and \eqref{eq:ee18btech11007_u_mat},
\begin{align}
\vec{I}-\vec{T} &= \myvec{1 & 0 & 0 & 1 & 1 \ \\
-1 & 1 & 0 & 0 & 0   \\
 0 & -s-1/s & 1 & 0 & 0 \\
 0 & 0 & -1 & 1 & 0 \\
0 & 0 & 0 & -1/s & 1 }
\\
\implies U_{40}&=\frac{\mydet{
-1&1&0&0 
\\0 &-s-1/s&1&0 
\\0&0&-1&1
\\0&0&0&-1/s}}
{
\myvec{1 & 0 & 0 & 1 & 1 \ \\
-1 & 1 & 0 & 0 & 0   \\
 0 & -s-1/s & 1 & 0 & 0 \\
 0 & 0 & -1 & 1 & 0 \\
0 & 0 & 0 & -1/s & 1 }
}
 \label{eq:ee18btech11007_u_gain},
\end{align}
%
using the cofactor expansion and  Problem \ref{prob:ee18btech11007_mat_form}. The gain is obtained as \eqref{eq:ee18btech11003_gain_sol} after expanding the determinants and simplifying.
%
\item Write a program to compute the gain using the matrix method.
\\
\solution The following code computes the transfer function 
\begin{lstlisting}
codes/ee18btech11007/MasonsGain.py
\end{lstlisting}
\end{enumerate}
