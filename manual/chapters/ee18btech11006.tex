\begin{enumerate}[label=\thesection.\arabic*.,ref=\thesection.\theenumi]
\numberwithin{equation}{enumi}
\item State the general model of a state space system specifying the dimensions of the matrices and vectors.
\\
\solution The model is given by 
\begin{align}
\dot{\vec{x}}(t)&=\vec{A}\vec{x}(t)+\vec{B}\vec{u}(t) \\
 \vec{y}(t)&=\vec{C}\vec{x}(t)+\vec{D} \vec{u}(t)
\label{eq:ee18btech11004_state}
\end{align}
with parameters listed in Table \ref{table:ee18btech11004}.
%
\begin{table}[!ht]
\centering
\begin{enumerate}[label=\thesubsection.\arabic*.,ref=\thesubsection.\theenumi]
\numberwithin{equation}{enumi}
\item State the general model of a state space system specifying the dimensions of the matrices and vectors.
\\
\solution The model is given by 
\begin{align}
\label{eq:ee18btech11004_state}
\dot{\vec{x}}(t)&=\vec{A}\vec{x}(t)+\vec{B}\vec{u}(t) \\
 \vec{y}(t)&=\vec{C}\vec{x}(t)+\vec{D} \vec{u}(t)
\end{align}
%
with parameters listed in Table \ref{table:ee18btech11004}.
%
\begin{table}[!ht]
\centering
\begin{enumerate}[label=\thesubsection.\arabic*.,ref=\thesubsection.\theenumi]
\numberwithin{equation}{enumi}
\item State the general model of a state space system specifying the dimensions of the matrices and vectors.
\\
\solution The model is given by 
\begin{align}
\label{eq:ee18btech11004_state}
\dot{\vec{x}}(t)&=\vec{A}\vec{x}(t)+\vec{B}\vec{u}(t) \\
 \vec{y}(t)&=\vec{C}\vec{x}(t)+\vec{D} \vec{u}(t)
\end{align}
%
with parameters listed in Table \ref{table:ee18btech11004}.
%
\begin{table}[!ht]
\centering
\begin{enumerate}[label=\thesubsection.\arabic*.,ref=\thesubsection.\theenumi]
\numberwithin{equation}{enumi}
\item State the general model of a state space system specifying the dimensions of the matrices and vectors.
\\
\solution The model is given by 
\begin{align}
\label{eq:ee18btech11004_state}
\dot{\vec{x}}(t)&=\vec{A}\vec{x}(t)+\vec{B}\vec{u}(t) \\
 \vec{y}(t)&=\vec{C}\vec{x}(t)+\vec{D} \vec{u}(t)
\end{align}
%
with parameters listed in Table \ref{table:ee18btech11004}.
%
\begin{table}[!ht]
\centering
\input{./tables/ee18btech11004.tex}
\caption{}
\label{table:ee18btech11004}
\end{table}
\item Find the transfer function $\vec{H}(s)$ for the general system.
\\
\solution 
Taking Laplace transform on both sides we have the following equations
\begin{align}
 s\vec{IX}(s)-\vec{x}(0)&= \vec{AX}(s)+ \vec{BU}(s)\\
(s\vec{I}-\vec{A})\vec{X}(s)&= \vec{BU}(s)+ \vec{x}(0)\\
\vec{X}(s)&={(s\vec{I}-\vec{A})^{-1}}\vec{B U}(s)\\
& +(s\vec{I}-\vec{A})^{-1}\vec{x}(0)
\label{eq:x_init}
\end{align}
and
\begin{align}
\vec{Y}(s)&= \vec{CX}(s)+D\vec{IU}(s)
\end{align}
Substituting from \eqref{eq:x_init} in the above,
%
\begin{multline}
\vec{Y}(s)=( \vec{C}{(s\vec{I}-\vec{A})^{-1}}\vec{B}+D\vec{I}) \vec{U}(s) 
\\
+ \vec{C}(s\vec{I}-\vec{A})^{-1}\vec{x}(0)
\end{multline}
%
%
\item Find $H(s)$ for a SISO (single input single output) system.
\\
\solution
\begin{align}
\label{eq:ee18btech11004_siso}
H(s)= {\frac{Y(s)}{U(s)}}= C{(sI-A)^{-1}}B+DI
\end{align}

\item Given 
\begin{align}
H(s)&=\frac{1}{s^3+3s^2+2s+1}
\\
D&=0
\\
\vec{B}&= \myvec{0\\0\\1}
\end{align}
%
 find $\vec{A}$ and $\vec{C}$ such that the state-space realization is in {\em controllable canonical form}.
\\
\solution 
\begin{align} 
\because {\frac{Y(s)}{U(s)}}= \frac{Y(s)}{V(s)} \times \frac{V(s)}{U(s)},
\end{align}
letting
\begin{align}
 {\frac{Y(s)}{V(s)}}= 1, 
\end{align}
results in 
\begin{align}
{\frac{U(s)}{V(s)}}={s^3 + 3s^2+2s + 1}
\end{align}

giving
\begin{align}
U(s)= s^3 V(s) + 3s^2 V(s)+2sV(s) + V(s)
\end{align}

so equation 0.1.13 can be written as
\begin{align}
\myvec{sV(s)\\s^2V(s)\\s^3V(s)}
=
\myvec{0&1&0\\0&0&1\\-1&-2&-3}\myvec{V(s)\\s(s)\\s^2V(s)}
+
\myvec{0\\0\\1}  U
\end{align}
So 
\begin{align}
\vec{A}=\myvec{0&1&0\\0&0&1\\-1&-2&-3}
\end{align}

\begin{align}
Y=X_{1}(s)
=\myvec{1&0&0} \myvec{V(s)\\sV(s)\\s^2V(s)}
\end{align}
\begin{align}
\vec{C}=\myvec{1&0&0}
\end{align}

\item Obtain $\vec{A}$ and $\vec{C}$ so that the state-space realization in in {\em observable canonical form}.
\\
\solution  Given that
\begin{align}
H(s)&=\frac{1}{s^3+3s^2+2s+1}
\end{align}
\begin{align}
\frac{Y(s)}{U(s)}=\frac{1}{s^3+3s^2+2s+1} \\
Y(s) \times (s^3+3s^2+2s+1) = U(s)
\end{align}
\begin{align}
s^3Y(s)+3s^2Y(s)+2sY(s)+Y(s)=U(s)\\
s^3Y(s)=U(s)-3s^2Y(s)-2sY(s)-Y(s)\\
Y(s)=-3s^{-1}Y(s)-2s^{-2}Y(s)+s^{-3}(U(s)-Y(s))
\end{align}
\\ let $Y=aU+X_{1}$
\\ by comparing with equation 1.5.6 we get a=0 and
\begin{align}
Y=X_{1}
\end{align}
inverse laplace transform of above equation is 
\begin{align}
y=x_{1}
\end{align}
so from above equation 1.5.6 and 1.5.7
\begin{align}
X_{1}=-3s^{-1}Y(s)-2s^{-2}Y(s)+s^{-3}(U(s)-Y(s))\\
sX_{1}=-3Y(s)-2s^{-1}Y(s)+s^{-2}(U(s)-Y(s)) 
\end{align}
inverse laplace transform of above equation 
\begin{align}
\dot{x_{1}}=-3y+x_{2}
\end{align} 
where
\begin{align}
X_{2}=-2s^{-1}Y(s)+s^{-2}(U(s)-Y(s))\\
sX_{2}=-2Y(s)+s^{-1}(U(s)-Y(s))
\end{align} 
inverse laplace transform of above equation 
\begin{align}
\dot{x_{2}}=-2y+x_{3}
\end{align}
where
\begin{align}
X_{3}=s^{-1}(U(s)-Y(s))\\
sX_{3}=U(s)-Y(s)
\end{align} 
inverse laplace transform of above equation 
\begin{align}
\dot{x_{3}}=u-y
\end{align}
so we get four equations which are
\begin{align}
y=x_{1}\\
\dot{x_{1}}=-3y+x_{2}\\
\dot{x_{2}}=-2y+x_{3}\\
\dot{x_{3}}=u-y
\end{align} 
sub $ y=x_{1}$ in 1.5.19,1.5.20,1.5.21 we get
\begin{align}
 y=x_{1}\\
\dot{x_{1}}=-3x_{1}+x_{2}\\
\dot{x_{2}}=-2x_{1}+x_{3}\\
\dot{x_{3}}=u-x_{1}
\end{align} 
so above equations can be written as
\begin{align}
\myvec{\dot{x_{1}}\\\dot{x_{2}}\\\dot{x_{3}})}
=
\myvec{-3&1&0\\-2&0&1\\-1&0&0}\myvec{x_{1}\\x_{2}\\x_{3}}
+
\myvec{0\\0\\1}  U
\end{align}
So 
\begin{align}
\vec{A}=\myvec{-3&1&0\\-2&0&1\\-1&0&0}
\end{align}
\begin{align}
y=x_{1}
=\myvec{1&0&0} \myvec{x_{1}\\x_{2}\\x_{3}}
\end{align}
\begin{align}
\vec{C}=\myvec{1&0&0}
\end{align}


\item Find the eigenvaues of $\vec{A}$ and the poles of $H(s)$ using a python code.
\\
\solution The following code 
%
\begin{lstlisting}
codes/ee18btech11004.py
\end{lstlisting}
gives the necessary values.  The roots are the same as the eigenvalues.
%
\item Theoretically, show that eigenvaues of $\vec{A}$ are the poles of  $H(s)$.
\solution 
\\ as we know tthat  the characteristic equation is det(sI-A) 
\\\begin{align}
\vec{sI-A}=
\myvec{s&0&0\\0&s&0\\0&0&s}
-
\myvec{0&1&0\\0&0&1\\-1&-2&-3}
=\myvec{s&-1&0\\0&s&-1\\1&2&s+3}
\end{align}
\\therfore
\begin{align}
det(sI-A)=s(s^2+3s+2)+1(1)=s^3+3s^2+2s+1
\end{align} 
\\so from equation 1.6.2 we can see that charcteristic equation is equal to the denominator of the transefer function
\end{enumerate}


\caption{}
\label{table:ee18btech11004}
\end{table}
\item Find the transfer function $\vec{H}(s)$ for the general system.
\\
\solution 
Taking Laplace transform on both sides we have the following equations
\begin{align}
 s\vec{IX}(s)-\vec{x}(0)&= \vec{AX}(s)+ \vec{BU}(s)\\
(s\vec{I}-\vec{A})\vec{X}(s)&= \vec{BU}(s)+ \vec{x}(0)\\
\vec{X}(s)&={(s\vec{I}-\vec{A})^{-1}}\vec{B U}(s)\\
& +(s\vec{I}-\vec{A})^{-1}\vec{x}(0)
\label{eq:x_init}
\end{align}
and
\begin{align}
\vec{Y}(s)&= \vec{CX}(s)+D\vec{IU}(s)
\end{align}
Substituting from \eqref{eq:x_init} in the above,
%
\begin{multline}
\vec{Y}(s)=( \vec{C}{(s\vec{I}-\vec{A})^{-1}}\vec{B}+D\vec{I}) \vec{U}(s) 
\\
+ \vec{C}(s\vec{I}-\vec{A})^{-1}\vec{x}(0)
\end{multline}
%
%
\item Find $H(s)$ for a SISO (single input single output) system.
\\
\solution
\begin{align}
\label{eq:ee18btech11004_siso}
H(s)= {\frac{Y(s)}{U(s)}}= C{(sI-A)^{-1}}B+DI
\end{align}

\item Given 
\begin{align}
H(s)&=\frac{1}{s^3+3s^2+2s+1}
\\
D&=0
\\
\vec{B}&= \myvec{0\\0\\1}
\end{align}
%
 find $\vec{A}$ and $\vec{C}$ such that the state-space realization is in {\em controllable canonical form}.
\\
\solution 
\begin{align} 
\because {\frac{Y(s)}{U(s)}}= \frac{Y(s)}{V(s)} \times \frac{V(s)}{U(s)},
\end{align}
letting
\begin{align}
 {\frac{Y(s)}{V(s)}}= 1, 
\end{align}
results in 
\begin{align}
{\frac{U(s)}{V(s)}}={s^3 + 3s^2+2s + 1}
\end{align}

giving
\begin{align}
U(s)= s^3 V(s) + 3s^2 V(s)+2sV(s) + V(s)
\end{align}

so equation 0.1.13 can be written as
\begin{align}
\myvec{sV(s)\\s^2V(s)\\s^3V(s)}
=
\myvec{0&1&0\\0&0&1\\-1&-2&-3}\myvec{V(s)\\s(s)\\s^2V(s)}
+
\myvec{0\\0\\1}  U
\end{align}
So 
\begin{align}
\vec{A}=\myvec{0&1&0\\0&0&1\\-1&-2&-3}
\end{align}

\begin{align}
Y=X_{1}(s)
=\myvec{1&0&0} \myvec{V(s)\\sV(s)\\s^2V(s)}
\end{align}
\begin{align}
\vec{C}=\myvec{1&0&0}
\end{align}

\item Obtain $\vec{A}$ and $\vec{C}$ so that the state-space realization in in {\em observable canonical form}.
\\
\solution  Given that
\begin{align}
H(s)&=\frac{1}{s^3+3s^2+2s+1}
\end{align}
\begin{align}
\frac{Y(s)}{U(s)}=\frac{1}{s^3+3s^2+2s+1} \\
Y(s) \times (s^3+3s^2+2s+1) = U(s)
\end{align}
\begin{align}
s^3Y(s)+3s^2Y(s)+2sY(s)+Y(s)=U(s)\\
s^3Y(s)=U(s)-3s^2Y(s)-2sY(s)-Y(s)\\
Y(s)=-3s^{-1}Y(s)-2s^{-2}Y(s)+s^{-3}(U(s)-Y(s))
\end{align}
\\ let $Y=aU+X_{1}$
\\ by comparing with equation 1.5.6 we get a=0 and
\begin{align}
Y=X_{1}
\end{align}
inverse laplace transform of above equation is 
\begin{align}
y=x_{1}
\end{align}
so from above equation 1.5.6 and 1.5.7
\begin{align}
X_{1}=-3s^{-1}Y(s)-2s^{-2}Y(s)+s^{-3}(U(s)-Y(s))\\
sX_{1}=-3Y(s)-2s^{-1}Y(s)+s^{-2}(U(s)-Y(s)) 
\end{align}
inverse laplace transform of above equation 
\begin{align}
\dot{x_{1}}=-3y+x_{2}
\end{align} 
where
\begin{align}
X_{2}=-2s^{-1}Y(s)+s^{-2}(U(s)-Y(s))\\
sX_{2}=-2Y(s)+s^{-1}(U(s)-Y(s))
\end{align} 
inverse laplace transform of above equation 
\begin{align}
\dot{x_{2}}=-2y+x_{3}
\end{align}
where
\begin{align}
X_{3}=s^{-1}(U(s)-Y(s))\\
sX_{3}=U(s)-Y(s)
\end{align} 
inverse laplace transform of above equation 
\begin{align}
\dot{x_{3}}=u-y
\end{align}
so we get four equations which are
\begin{align}
y=x_{1}\\
\dot{x_{1}}=-3y+x_{2}\\
\dot{x_{2}}=-2y+x_{3}\\
\dot{x_{3}}=u-y
\end{align} 
sub $ y=x_{1}$ in 1.5.19,1.5.20,1.5.21 we get
\begin{align}
 y=x_{1}\\
\dot{x_{1}}=-3x_{1}+x_{2}\\
\dot{x_{2}}=-2x_{1}+x_{3}\\
\dot{x_{3}}=u-x_{1}
\end{align} 
so above equations can be written as
\begin{align}
\myvec{\dot{x_{1}}\\\dot{x_{2}}\\\dot{x_{3}})}
=
\myvec{-3&1&0\\-2&0&1\\-1&0&0}\myvec{x_{1}\\x_{2}\\x_{3}}
+
\myvec{0\\0\\1}  U
\end{align}
So 
\begin{align}
\vec{A}=\myvec{-3&1&0\\-2&0&1\\-1&0&0}
\end{align}
\begin{align}
y=x_{1}
=\myvec{1&0&0} \myvec{x_{1}\\x_{2}\\x_{3}}
\end{align}
\begin{align}
\vec{C}=\myvec{1&0&0}
\end{align}


\item Find the eigenvaues of $\vec{A}$ and the poles of $H(s)$ using a python code.
\\
\solution The following code 
%
\begin{lstlisting}
codes/ee18btech11004.py
\end{lstlisting}
gives the necessary values.  The roots are the same as the eigenvalues.
%
\item Theoretically, show that eigenvaues of $\vec{A}$ are the poles of  $H(s)$.
\solution 
\\ as we know tthat  the characteristic equation is det(sI-A) 
\\\begin{align}
\vec{sI-A}=
\myvec{s&0&0\\0&s&0\\0&0&s}
-
\myvec{0&1&0\\0&0&1\\-1&-2&-3}
=\myvec{s&-1&0\\0&s&-1\\1&2&s+3}
\end{align}
\\therfore
\begin{align}
det(sI-A)=s(s^2+3s+2)+1(1)=s^3+3s^2+2s+1
\end{align} 
\\so from equation 1.6.2 we can see that charcteristic equation is equal to the denominator of the transefer function
\end{enumerate}


\caption{}
\label{table:ee18btech11004}
\end{table}
\item Find the transfer function $\vec{H}(s)$ for the general system.
\\
\solution 
Taking Laplace transform on both sides we have the following equations
\begin{align}
 s\vec{IX}(s)-\vec{x}(0)&= \vec{AX}(s)+ \vec{BU}(s)\\
(s\vec{I}-\vec{A})\vec{X}(s)&= \vec{BU}(s)+ \vec{x}(0)\\
\vec{X}(s)&={(s\vec{I}-\vec{A})^{-1}}\vec{B U}(s)\\
& +(s\vec{I}-\vec{A})^{-1}\vec{x}(0)
\label{eq:x_init}
\end{align}
and
\begin{align}
\vec{Y}(s)&= \vec{CX}(s)+D\vec{IU}(s)
\end{align}
Substituting from \eqref{eq:x_init} in the above,
%
\begin{multline}
\vec{Y}(s)=( \vec{C}{(s\vec{I}-\vec{A})^{-1}}\vec{B}+D\vec{I}) \vec{U}(s) 
\\
+ \vec{C}(s\vec{I}-\vec{A})^{-1}\vec{x}(0)
\end{multline}
%
%
\item Find $H(s)$ for a SISO (single input single output) system.
\\
\solution
\begin{align}
\label{eq:ee18btech11004_siso}
H(s)= {\frac{Y(s)}{U(s)}}= C{(sI-A)^{-1}}B+DI
\end{align}

\item Given 
\begin{align}
H(s)&=\frac{1}{s^3+3s^2+2s+1}
\\
D&=0
\\
\vec{B}&= \myvec{0\\0\\1}
\end{align}
%
 find $\vec{A}$ and $\vec{C}$ such that the state-space realization is in {\em controllable canonical form}.
\\
\solution 
\begin{align} 
\because {\frac{Y(s)}{U(s)}}= \frac{Y(s)}{V(s)} \times \frac{V(s)}{U(s)},
\end{align}
letting
\begin{align}
 {\frac{Y(s)}{V(s)}}= 1, 
\end{align}
results in 
\begin{align}
{\frac{U(s)}{V(s)}}={s^3 + 3s^2+2s + 1}
\end{align}

giving
\begin{align}
U(s)= s^3 V(s) + 3s^2 V(s)+2sV(s) + V(s)
\end{align}

so equation 0.1.13 can be written as
\begin{align}
\myvec{sV(s)\\s^2V(s)\\s^3V(s)}
=
\myvec{0&1&0\\0&0&1\\-1&-2&-3}\myvec{V(s)\\s(s)\\s^2V(s)}
+
\myvec{0\\0\\1}  U
\end{align}
So 
\begin{align}
\vec{A}=\myvec{0&1&0\\0&0&1\\-1&-2&-3}
\end{align}

\begin{align}
Y=X_{1}(s)
=\myvec{1&0&0} \myvec{V(s)\\sV(s)\\s^2V(s)}
\end{align}
\begin{align}
\vec{C}=\myvec{1&0&0}
\end{align}

\item Obtain $\vec{A}$ and $\vec{C}$ so that the state-space realization in in {\em observable canonical form}.
\\
\solution  Given that
\begin{align}
H(s)&=\frac{1}{s^3+3s^2+2s+1}
\end{align}
\begin{align}
\frac{Y(s)}{U(s)}=\frac{1}{s^3+3s^2+2s+1} \\
Y(s) \times (s^3+3s^2+2s+1) = U(s)
\end{align}
\begin{align}
s^3Y(s)+3s^2Y(s)+2sY(s)+Y(s)=U(s)\\
s^3Y(s)=U(s)-3s^2Y(s)-2sY(s)-Y(s)\\
Y(s)=-3s^{-1}Y(s)-2s^{-2}Y(s)+s^{-3}(U(s)-Y(s))
\end{align}
\\ let $Y=aU+X_{1}$
\\ by comparing with equation 1.5.6 we get a=0 and
\begin{align}
Y=X_{1}
\end{align}
inverse laplace transform of above equation is 
\begin{align}
y=x_{1}
\end{align}
so from above equation 1.5.6 and 1.5.7
\begin{align}
X_{1}=-3s^{-1}Y(s)-2s^{-2}Y(s)+s^{-3}(U(s)-Y(s))\\
sX_{1}=-3Y(s)-2s^{-1}Y(s)+s^{-2}(U(s)-Y(s)) 
\end{align}
inverse laplace transform of above equation 
\begin{align}
\dot{x_{1}}=-3y+x_{2}
\end{align} 
where
\begin{align}
X_{2}=-2s^{-1}Y(s)+s^{-2}(U(s)-Y(s))\\
sX_{2}=-2Y(s)+s^{-1}(U(s)-Y(s))
\end{align} 
inverse laplace transform of above equation 
\begin{align}
\dot{x_{2}}=-2y+x_{3}
\end{align}
where
\begin{align}
X_{3}=s^{-1}(U(s)-Y(s))\\
sX_{3}=U(s)-Y(s)
\end{align} 
inverse laplace transform of above equation 
\begin{align}
\dot{x_{3}}=u-y
\end{align}
so we get four equations which are
\begin{align}
y=x_{1}\\
\dot{x_{1}}=-3y+x_{2}\\
\dot{x_{2}}=-2y+x_{3}\\
\dot{x_{3}}=u-y
\end{align} 
sub $ y=x_{1}$ in 1.5.19,1.5.20,1.5.21 we get
\begin{align}
 y=x_{1}\\
\dot{x_{1}}=-3x_{1}+x_{2}\\
\dot{x_{2}}=-2x_{1}+x_{3}\\
\dot{x_{3}}=u-x_{1}
\end{align} 
so above equations can be written as
\begin{align}
\myvec{\dot{x_{1}}\\\dot{x_{2}}\\\dot{x_{3}})}
=
\myvec{-3&1&0\\-2&0&1\\-1&0&0}\myvec{x_{1}\\x_{2}\\x_{3}}
+
\myvec{0\\0\\1}  U
\end{align}
So 
\begin{align}
\vec{A}=\myvec{-3&1&0\\-2&0&1\\-1&0&0}
\end{align}
\begin{align}
y=x_{1}
=\myvec{1&0&0} \myvec{x_{1}\\x_{2}\\x_{3}}
\end{align}
\begin{align}
\vec{C}=\myvec{1&0&0}
\end{align}


\item Find the eigenvaues of $\vec{A}$ and the poles of $H(s)$ using a python code.
\\
\solution The following code 
%
\begin{lstlisting}
codes/ee18btech11004.py
\end{lstlisting}
gives the necessary values.  The roots are the same as the eigenvalues.
%
\item Theoretically, show that eigenvaues of $\vec{A}$ are the poles of  $H(s)$.
\solution 
\\ as we know tthat  the characteristic equation is det(sI-A) 
\\\begin{align}
\vec{sI-A}=
\myvec{s&0&0\\0&s&0\\0&0&s}
-
\myvec{0&1&0\\0&0&1\\-1&-2&-3}
=\myvec{s&-1&0\\0&s&-1\\1&2&s+3}
\end{align}
\\therfore
\begin{align}
det(sI-A)=s(s^2+3s+2)+1(1)=s^3+3s^2+2s+1
\end{align} 
\\so from equation 1.6.2 we can see that charcteristic equation is equal to the denominator of the transefer function
\end{enumerate}


\caption{}
\label{table:ee18btech11004}
\end{table}

\item Find the transfer function $\vec{H}(s)$ for the general system.
\\
\solution 
Taking Laplace transform on both sides we have the following equations
\begin{align}
 s\vec{IX}(s)-\vec{x}(0)&= \vec{AX}(s)+ \vec{BU}(s)\\
(s\vec{I}-\vec{A})\vec{X}(s)&= \vec{BU}(s)+ \vec{x}(0)\\
\vec{X}(s)&={(s\vec{I}-\vec{A})^{-1}}\vec{B U}(s)\\
& +(s\vec{I}-\vec{A})^{-1}\vec{x}(0)
\label{eq:ee18btech11006_x_init}
\end{align}
and
\begin{align}
\vec{Y}(s)&= \vec{C}\vec{X}(s)+\vec{D}\vec{U}(s)
\end{align}
Substituting from \eqref{eq:x_init} in the above,
%
\begin{multline}
\vec{Y}(s)=( \vec{C}{(s\vec{I}-\vec{A})^{-1}}\vec{B}+\vec{D}\vec{I}) \vec{U}(s) 
\\
+ \vec{C}(s\vec{I}-\vec{A})^{-1}\vec{x}(0)
\label{eq:ee18btech11006_yx_init}
\end{multline}
%
\item Find $H(s)$ for a SISO (single input single output) system.
\\
\solution
\begin{align}
\label{eq:ee18btech11004_siso}
H(s)= {\frac{Y(s)}{U(s)}}= \vec{C}{(s\vec{I}-\vec{A})^{-1}}\vec{B}+D\vec{I}
\end{align}

\item Given 
\begin{align}
\label{eq:ee18btech11004_system}
H(s)&=\frac{1}{s^3+3s^2+2s+1}
\\
D&=0
\\
\vec{B}&= \myvec{0\\0\\1}
\end{align}
%
 find $\vec{A}$ and $\vec{C}$ such that the state-space realization is in {\em controllable canonical form}.
\\
\solution 
\begin{align} 
\because {\frac{Y(s)}{U(s)}}= \frac{Y(s)}{V(s)} \times \frac{V(s)}{U(s)},
\end{align}
letting
\begin{align}
 {\frac{Y(s)}{V(s)}}= 1, 
\end{align}
results in 
\begin{align}
{\frac{U(s)}{V(s)}}={s^3 + 3s^2+2s + 1}
\end{align}

giving
\begin{align}
U(s)= s^3 V(s) + 3s^2 V(s)+2sV(s) + V(s)
\end{align}
%
so the above equation  can be written as
\begin{align}
\myvec{sV(s)\\s^2V(s)\\s^3V(s)}
=
\myvec{0&1&0\\0&0&1\\-1&-2&-3}\myvec{V(s)\\sV(s)\\s^2V(s)}
+
\myvec{0\\0\\1}  U
\end{align}
Letting
\begin{align}
\vec{A}&=\myvec{0&1&0\\0&0&1\\-1&-2&-3}
\\
\vec{X}_1 &= \myvec{sV(s)\\s^2V(s)\\s^3V(s)}
\\
\vec{X} &= \myvec{V(s)\\sV(s)\\s^2V(s)},
\end{align}
\\
\begin{align}
\vec{X}_{1}(s) &= \vec{A}\vec{X}(s)+ \vec{B}U(s)
\\
Y&=\vec{C}\vec{X}_{1}(s)
\end{align}
where
\begin{align}
\vec{C}=\myvec{1&0&0}
\end{align}

\item Obtain $\vec{A}$ and $\vec{C}$ so that the state-space realization in in {\em observable canonical form}.
\\
\solution  Given that
\begin{align}
H(s)&=\frac{1}{s^3+3s^2+2s+1},
\\
%\end{align}
%\begin{align}
\frac{Y(s)}{U(s)}&=\frac{1}{s^3+3s^2+2s+1} \\
\implies  U(s)&=Y(s)  (s^3+3s^2+2s+1)
%\end{align}
%\begin{align}
%U(s)&=s^3Y(s)+3s^2Y(s)+2sY(s)+Y(s)\\
%s^3Y(s)&=U(s)-3s^2Y(s)-2sY(s)-Y(s)\\
\\
\text{or, }Y(s)&=-3s^{-1}Y(s)-2s^{-2}Y(s)+
\nonumber \\
&\quad s^{-3}(U(s)-Y(s))
\end{align}
%\\ let $Y=aU+X_{1}$
%\\ by comparing with equation 1.5.6 we get a=0 and
%so from above equation 1.5.6 and 1.5.7
Let
\begin{align}
X_{1}(s)&=Y(s) = -3s^{-1}Y(s)-2s^{-2}Y(s) \nonumber \\
&\quad +s^{-3}(U(s)-Y(s))\\
%sX_{1}(s)&=-3Y(s)-2s^{-1}Y(s)\\ 
%&+s^{-2}(U(s)-Y(s))
%\end{align} 
%\begin{align}
%sX_{1}(s)&=-3Y(s)+X_{2}(s)
%\end{align} 
%where
%\begin{align}
X_{2}(s)&=-2s^{-1}Y(s)+s^{-2}(U(s)-Y(s))\\
%sX_{2}(s)&=-2Y(s)+s^{-1}(U(s)-Y(s))
%\end{align} 
%\begin{align}
%sX_{2}(s)&=-2Y(s)+X_{3}(s)
%\end{align}
%where
%\begin{align}
X_{3}(s)&=s^{-1}(U(s)-Y(s))
%sX_{3}(s)&=U(s)-Y(s)
\end{align} 
%\begin{align}
%X_{3}(s)&=U(s)-Y(s)
%\end{align}
\begin{align}
\implies
\begin{split}
sX_{1}(s)&=-3Y(s)+X_{2}(s)\\
sX_{2}(s)&=-2Y(s)+X_{3}(s)\\
sX_{3}(s)&=U(s)-Y(s)
\end{split}
\end{align} 
Substituting  $ Y=X_{1}(s)$ the above, 
\begin{align}
sX_{1}(s)&=-3X_{1}(s)+X_{2}(s)\\
sX_{2}(s)&=-2X_{1}(s)+X_{3}(s)\\
sX_{3}(s)&=U(s)-X_{1}(s)
\end{align} 
which can be expressed as
\begin{align}
\myvec{sX_{1}(s)\\sX_{2}(s)\\sX_{3}(s)}
&=
\myvec{-3&1&0\\-2&0&1\\-1&0&0}\myvec{X_{1}(s)\\X_{2}(s)\\X_{3}(s)}
+
\myvec{0\\0\\1}  U
\\
\text{or, }
\begin{split}
s\vec{X}(s) &= \vec{A}\vec{X}(s) + \vec{B}U(s)
\\
Y(s) &= \vec{B}\vec{X}(s)
\end{split}
\end{align}
where
\begin{align}
\vec{A}&=\myvec{-3&1&0\\-2&0&1\\-1&0&0}
\\
\vec{B}&= \myvec{1&0&0}
\end{align}
%\begin{align}
%Y(s)=X_{1}(s)
%=\myvec{1&0&0} \myvec{X_{1}(s)\\X_{2}(s)\\X_{3}(s)}
%\end{align}
%\begin{align}
%\vec{C}&=\myvec{1&0&0}
%\end{align}
%
%
%
%\\
%\solution  Given that
%\begin{align}
%H(s)&=\frac{1}{s^3+3s^2+2s+1},
%\\
%\frac{Y(s)}{U(s)}&=\frac{1}{s^3+3s^2+2s+1} \\
%%\implies  U(s)&= (s^3+3s^2+2s+1)Y(s) 
%%\\
%%\implies 
%%U(s)&=s^3Y(s)+3s^2Y(s)+2sY(s)+Y(s)\\
%%s^3Y(s)&=U(s)-3s^2Y(s)-2sY(s)-Y(s)\\
%\implies Y(s)&=-3s^{-1}Y(s)-2s^{-2}Y(s) \nonumber \\
%&\quad +s^{-3}\brak{U(s)-Y(s)}
%\end{align}
%%
%after some algebra.
%\\ let $Y=aU+X_{1}$
%\\ by comparing with equation 1.5.6 we get a=0 and
%\begin{align}
%Y=X_{1}
%\end{align}
%inverse laplace transform of above equation is 
%\begin{align}
%y=x_{1}
%\end{align}
%so from above equation 1.5.6 and 1.5.7
%\begin{align}
%X_{1}&=-3s^{-1}Y(s)-2s^{-2}Y(s)+s^{-3}(U(s)-Y(s))\\
%sX_{1}&=-3Y(s)-2s^{-1}Y(s)+s^{-2}(U(s)-Y(s)) 
%\end{align}
%inverse laplace transform of above equation 
%\begin{align}
%\dot{x_{1}}&=-3y+x_{2}
%\end{align} 
%where
%\begin{align}
%X_{2}&=-2s^{-1}Y(s)+s^{-2}(U(s)-Y(s))\\
%sX_{2}&=-2Y(s)+s^{-1}(U(s)-Y(s))
%\end{align} 
%inverse laplace transform of above equation 
%\begin{align}
%\dot{x_{2}}&=-2y+x_{3}
%\end{align}
%where
%\begin{align}
%X_{3}&=s^{-1}(U(s)-Y(s))\\
%sX_{3}&=U(s)-Y(s)
%\end{align} 
%inverse laplace transform of above equation 
%\begin{align}
%\dot{x_{3}}&=u-y
%\end{align}
%so we get four equations which are
%\begin{align}
%x_{1}&=y\\
%\dot{x_{1}}&=-3y+x_{2}\\
%\dot{x_{2}}&=-2y+x_{3}\\
%\dot{x_{3}}&=u-y
%\end{align} 
%sub $ y=x_{1}$ in 1.5.19,1.5.20,1.5.21 we get
%\begin{align}
%x_{1}&=y\\
%\dot{x_{1}}&=-3x_{1}+x_{2}\\
%\dot{x_{2}}&=-2x_{1}+x_{3}\\
%\dot{x_{3}}&=u-x_{1}
%\end{align} 
%so above equations can be written as
%\begin{align}
%\myvec{\dot{x_{1}}\\\dot{x_{2}}\\\dot{x_{3}})}
%=
%\myvec{-3&1&0\\-2&0&1\\-1&0&0}\myvec{x_{1}\\x_{2}\\x_{3}}
%+
%\myvec{0\\0\\1}  U
%\end{align}
%So 
%\begin{align}
%\vec{A}=\myvec{-3&1&0\\-2&0&1\\-1&0&0}
%\end{align}
%\begin{align}
%y=x_{1}
%=\myvec{1&0&0} \myvec{x_{1}\\x_{2}\\x_{3}}
%\end{align}
%\begin{align}
%\vec{C}&=\myvec{1&0&0}
%\end{align}
%

\item Find the eigenvaues of $\vec{A}$ and the poles of $H(s)$ using a python code.
\\
\solution The following code 
%
\begin{lstlisting}
codes/ee18btech11004.py
\end{lstlisting}
gives the necessary values.  The roots are the same as the eigenvalues.
%
\item Theoretically, show that eigenvaues of $\vec{A}$ are the poles of  $H(s)$.\\
\solution 
As we know that  the characteristic equation is det(sI-A) 
\\\begin{align}
s\vec{I}-\vec{A}=
\myvec{s&0&0\\0&s&0\\0&0&s}
-
\myvec{0&1&0\\0&0&1\\-1&-2&-3}\\
=\myvec{s&-1&0\\0&s&-1\\1&2&s+3}
\end{align}

\begin{align}
\implies \mydet{s\vec{I}-\vec{A}}&=s(s^2+3s+2)+1(1)\\
&=s^3+3s^2+2s+1
\end{align} 
which is the denominator of $H(s)$ in \eqref{eq:ee18btech11004_system}
%
\end{enumerate}

