\begin{enumerate}[label=\thesection.\arabic*.,ref=\thesection.\theenumi]
\numberwithin{equation}{enumi} 
\item
 
(\textbf{a}). The open loop transfer function of a feedback control system is  
\begin{align}
G(s) = \frac{1}{s(1+2s)(1+s)} 
\end{align}
Find the gain margin of this system and analyse the stability.
(\textbf{b}). Provide some other example other than this to analyse stability mathematically using the concept of gain margin.
 \\



\solution
\\
(\textbf{a}) : \textbf{Gain Margin}:The greater the Gain Margin (GM), the greater the stability of the system. The gain margin refers to the amount of gain, which can be increased or decreased without making the system unstable. It is usually expressed as a magnitude in dB.
\\

We can usually read the gain margin directly from the Bode plot. This is done by calculating the vertical distance between the magnitude curve (on the Bode magnitude plot) and the x-axis at the frequency where the Bode phase plot = 180$^{\circ}$. This point is known as the phase crossover frequency.

Gain Margin is given by,
\begin{align}
G.M = -20log_{10}|G(j\omega_{pc})| = 20log_{10}k_{g}
\end{align}
where 
\begin{align}
k_{g}=\frac{1}{|G(j\omega_{pc})|} 
\end{align}
\\

Now let's put s = j$\omega$ in the equation of G(s) :
\begin{align}
G(j\omega) = \frac{1}{j\omega(1+2j\omega)(1+j\omega)} 
\end{align}
So,
\begin{align}
G(j\omega) = \frac{1}{j\omega(1+3j\omega-2\omega^2)}=\frac{1}{j\omega-3\omega^2-2j\omega^3}
\end{align}
Hence ,
\begin{align}
G(j\omega) = \frac{1}{-3\omega^2+j\omega(1-2\omega^2)} 
\end{align}

Now we know that $\omega_{pc}$ is the Phase crossover frequency (The frequency at which the phase of open-loop transfer function reaches -180$^{\circ}$ or +180$^{\circ}$ depending upon the range of tan inverse function).\\
Now,
\begin{align}
\angle G(j\omega)=- tan^{-1}(\frac{\omega(1-2\omega^2)}{-3\omega^2})
\end{align}
So,at $\omega=\omega_{pc}$ :
\begin{align}
\omega(1-2\omega^2) = 0 
\end{align}
i.e. the imaginary part of G(j$\omega$) = 0.So ,
\begin{align}
\omega_{pc} = \frac{1}{\sqrt{2}} 
\end{align}
as $\omega_{pc}$ should be positive and $\omega_{pc}$ shuld not be equal to zero.So now G(j$\omega_{pc}$) will be :
\begin{align}
G(j\omega_{pc}) = \frac{1}{-3\omega_{pc}^2}
\end{align}
i.e,
\begin{align}
|G(j\omega_{pc})| = \frac{1}{(\frac{3}{2})}
\end{align}
\begin{align}
k_{g}=\frac{1}{|G(j\omega_{pc})|} = \frac{3}{2}=1.5
\end{align}
So , Gain margin in terms of dB is :

\begin{align}
20log_{10}1.5 = 3.5dB
\end{align}
\\

Plot obtained for verification in python :

(You can download code from codes/ee18btech11016.py)
\begin{figure}[htp]
	\centering
	\includegraphics[width=\columnwidth,scale=2]{./figs/ee18btech11016/fig.eps}
	\caption{}
	\label{fig:Phase}
\end{figure}
\\

\section{Stability}

So,in the above figure, since $20log_{10}(G(j\omega_{pc}))$ = -3.5dB at $\omega_{pc} = -180^{\circ}$ so G.M = +3.5dB And since the gain margin is positive we can say that the system is stable more precisely the system is marginally stable as one of the pole lies on the imaginary axis.(Because for stability , both gain and phase margin should be positive.)
\\


(\textbf{b}).
\\

(\textbf{1})Gain of closed loop transfer function would not affect its stability.
\\\

(\textbf{2})Gain of open loop transfer function would affect the closed loop stability.
\\

The idea is very simple. When you derive the unit feedback closed loop transfer function of an open loop transfer function, it will be very obvious. Let's analyze it over a simple example;
\\

G = K/(s+1)
Unit Feedback Transfer Function = G/(1+G) = K/(s+1+K)
\\

As you can see from UFTF, open loop gain K affects both gain of the closed loop system and the pole location of it, thus its stability. And therefore if we increase the gain above  the gain margin then the system will become unstable.The pole location of closed loop system is (-1-K) and as long as this term is not positive, the system will be stable.


\end{enumerate}


